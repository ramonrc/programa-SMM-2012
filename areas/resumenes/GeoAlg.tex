\section{Geometr\'\i a Algebraica}

%%%%%%%%%%%%1146%%%%%%%%%%%
\subsection{\sffamily Geometr\'\i a de residuos {\footnotesize (CPI, 2Lic)}} \label{reg-1146} \index{mucino raymundo jesus r.@Muci�o Raymundo Jes�s R.!\ref{reg-1146}}
\noindent {\bfseries Jes\'us  Muci\~no Raymundo}, {\tt muciray@matmor.unam.mx}  {\slshape (Centro de Ciencias Matem\'aticas (UNAM))}\\
          \noindent  Gracias a los trabajos de Abel, Riemann, Jacobi, Torelli; los periodos de una superficie de Riemann o curva algebraica (integrales de 1--formas holomorfas)  describen a la curva sin ambig$\ddot{\hbox{u}}$edad alguna. Mostramos qu\'e se puede esperar para 1--formas meromorfas.
%%%%%%%%%%%%628%%%%%%%%%%%
\subsection{\sffamily Mapeos racionales del espacio proyectivo {\footnotesize (CI, Pos)}} \label{reg-628} \index{vargas mendoza jose antonio@Vargas Mendoza Jos\'e Antonio!\ref{reg-628}}
\noindent {\bfseries Jos\'e Antonio Vargas Mendoza}, {\tt javargas1@excite.com}  {\slshape (Instituto Polit\'ecnico Nacional (IPN)CIIDIR-OAXACA)}\\
          \noindent We construct rational maps of $ \mathbb{P}^n $ which have a prescribed variety as a component of its fixed point set. Our methods are focused on associated matrices of forms of constant degree; and the ``triple action of $ G = PGL_{n + 1} $ on them''. We include a complete classification of these maps and matrices for the case of the smooth conic curve in $ \mathbb{P}^2 $; and we also study the dynamical systems obtained by the iteration of these maps. We obtain invariants and canonical forms for the orbits of our matrices (modulo some syzygies) and also for the orbits of their associated maps under conjugation by $G$.
%%%%%%%%%%%%1022%%%%%%%%%%%
\subsection{\sffamily Geometr\'\i a algebraica a trav\'es de ejemplos {\footnotesize (CDV, 2Lic)}} \label{reg-1022} \index{elizondo huerta enrique javier@Elizondo Huerta Enrique Javier!\ref{reg-1022}}
\noindent {\bfseries Enrique Javier Elizondo Huerta}, {\tt javier@math.unam.mx}  {\slshape (Instituto de Matem\'aticas, UNAM)}\\
          \noindent En esta pl\'atica se ver\'an, a trav\'es de ejemplos, algunos de los problemas que se investigan en geometr\'\i a algebraica. Pretende responder en forma parcial la pregunta ?`qu\'e es geometr\'\i a algebraica? Es una pl\'atica dirigida a estudiantes de la carrera de matem\'aticas y s\'olo se requiere un poco de \'algebra. Tambi\'en es para profesores e investigadores interesados pero que trabajan en otras  \'areas de matem\'aticas.
%%%%%%%%%%%%1294%%%%%%%%%%%
\subsection{\sffamily La funci\'on Zeta mot\'\i vica {\footnotesize (CI, Pos)}} \label{reg-1294} \index{pompeyo gutierrez carlos ariel@Pompeyo Guti\'errez Carlos Ariel!\ref{reg-1294}}
\noindent {\bfseries Carlos Ariel Pompeyo Guti\'errez}, {\tt cariem200x@gmail.com}  {\slshape (Mathematics Department Texas A\&M University (TAMU))}\\
        \noindent Supongamos que $k=\mathbb{F}_{q}$ es un campo finito con $q$ elementos; $k_{r}=\mathbb{F}_{q^{r}}$ es una extensi\'on finita de $k$ y denotemos por $\bar{k}$ a una cerradura algebraica de $k$. Para una variedad algebraica $X$ definida sobre $k$, la funci\'on zeta de Hasse-Weil se define como\[ Z(X,t)= \exp \left( \sum_{r=1}^{\infty} N_{r} \frac{t^{r}}{r}  \right) \ ;  \]donde $N_{r}$ es el n\'umero de puntos $k_{r}$-racionales de $\bar{X}:= X \times_{k} \bar{k}$. Por definici\'on, $Z(X,t)$ es una serie formal de potencias con coeficientes racionales. Usando potencias sim\'etricas de la variedad $X$, la funci\'on zeta de Hasse-Weil puede reescribirse como\[ Z(X,t)= \sum_{r=0}^{\infty} \# Sym^{r}(X)(k) t^{r} \ . \]Esta \'ultima expresi\'on permiti\'o a Kapranov generalizar dicha funci\'on zeta en el contexto del anillo de variedades algebraicas definidas sobre un campo $F$, $K_{0}(Var(F))$, proponiendo la funci\'on zeta\[ Z_{\mu}(X,t)= \sum_{r=0}^{\infty} \mu[Sym^{r}(X)] t^{r} \ ; \]donde $\mu:K_{0}(Var(F)) \to R$ es un homomorfismo arbitrario de anillos. Cuando $F= \mathbb{Q}$, se puede reducir $X$ m\'odulo un primo $p$ y contar los puntos en $\mathbb{F}_{p}$ de la reducci\'on; en este caso, la funci\'on zeta de Kapranov se especializa a la funci\'on zeta de Hasse-Weil de la reducci\'on, por lo tanto puede considerarse como una interpolaci\'on de las funciones zeta de Hasse-Weil cuando $p$ var\'ia.Si $M$ es un motivo sobre $k$ (con coeficientes racionales), la funci\'on zeta mot\'ivica\[ Z_{mot}(M,t)= \sum_{r=0}^{\infty} [Sym^{r}(M)] t^{r} \in K_{0}(M_{Rat}(F))[[t]] \]se introduce para tratar de solucionar algunos problemas de racionalidad que tiene la funci\'on zeta de Kapranov. Se cree, por ejemplo, que $Z_{mot}(M,t)$ es racional para todo $M$ que provenga de una una variedad lisa. En esta pl\'atica presentaremos con m\'as detalle estas construcciones y discutiremos el estado de la conjetura de racionalidad de la funci\'on zeta mot\'ivica.
%%%%%%%%%%%%489%%%%%%%%%%%
\subsection{\sffamily Funciones Zeta de polinomios de Laurent sobre cuerpos $p-$\'adicos {\footnotesize (CI, Pos)}} \label{reg-489} \index{leon cardenal edwin@Le�n Cardenal Edwin!\ref{reg-489}}
\noindent {\bfseries Edwin  Le\'on Cardenal}, {\tt eleon@math.cinvestav.mx}  {\slshape (Centro de Investigaci\'on y Estudios Avanzados del Instituto Polit\'ecnico Nacional)}\\
          \noindent En esta charla introducimos un nuevo tipo de funciones zeta locales para polinomios de Laurent en dos variables sobre cuerpos $p-$\'adicos. Sea $K$ un cuerpo $p-$\'adico y sea $f(x_1,x_2)\in K[x_1,x_2,x_1^{-1},x_2^{-1}]$. Sea $\Phi$ una funci\'on de Bruhat-Schwartz, y tomemos $\omega$ un homomorfismo continuo de $K^{\times}$ en $\mathbb{C}^{\times}$. A estos datos podemos asociar la siguiente funci\'on zeta local:\[Z_{\Phi}\left(  \omega,f\right)  :=Z_{\Phi}\left(  s,\chi,f\right)  =\int\limits_{T^{2}\left(  K\right)  }\Phi\left(  x\right)  \omega\left(  f\left(  x\right)  \right)  \left\vert dx\right\vert ,\]donde $T^2(K)$ es el toro $2-$dimensional y $\left\vert dx\right\vert $ es la medida de Haar normalizada de $K^{2}$. Usando resoluci\'on t\'orica de singularidades mostramos la existencia de una continuaci\'on merom\'orfica para $Z_{\Phi}\left(  \omega,f\right)$ como funci\'on racional de $q^{-s}$. En contraste con las cl\'asicas funciones zeta de Igusa [2], la continuaci\'on merom\'orfica de $Z_{\Phi}\left(  \omega,f\right)$ tiene polos con partes reales positivas y negativas. Estas funciones zeta controlan el comportamiento asint\'otico de ciertas integrales oscilatorias del tipo\[E_{\Phi}\left(  z,f\right)  =\int\limits_{\left(  \mathbb{Q}_{p}^{\times}\right)  ^{2}}\Phi\left(  x_1,x_2\right)  \Psi\left(  zf\left(  x_1,x_2\right)  \right)dx_1\wedge dx_2,\]donde $\Psi$ es un car\'acter aditivo fijo de $\mathbb{Q}_{p}$, $z=up^{-m}$, con $u\in\mathbb{Z}_{p}^{\times}$, y$m\in\mathbb{Z}$. Un caso particular de estas integrales oscilatorias son las sumas exponenciales\[S_{m}=\sum\limits_{(x_1,x_2)\in\left(  \mathbb{Z}^{\times}/p^{m}\mathbb{Z}\right) \times\left(  \mathbb{Z}/p^{m}\mathbb{Z}\right)}e^{\frac{2\pi i}{p^m}\left( f(x_1,x_2)\right)}.\]Esta charla es fruto del trabajo conjunto con el Dr. Wilson Z\'u\~niga, ver [6]. Bibliograf\'\i a: [1] {\sc Denef J., Sperber S.},{\it Exponen\-tial sums mod $p^{n}$ and Newton polyhedra.} A tribute to Maurice Boffa. Bull. Belg. Math. Soc. Simon Stevin 2001, suppl., 55--63. [2] \textsc{Igusa J.--I.}, \textit{An Introduction to the Theory of Local Zeta Functions.}  AMS/IP Studies in Advanced Mathematics vol. 14, Amer. Math. Soc., Providence, RI, 2000. [3] {\sc Khovanskii A. G.}, {\it Newton polyhedra (resolution of singularities)}. (Russian) Current problems in mathematics, Vol. 22, 207--239, Itogi Nauki i Tekhniki, Akad. Nauk SSSR, Vsesoyuz. Inst. Nauchn. i Tekhn.Inform., Moscow, 1983. [4] {\sc Varchenko A.}, \textit{Newton polyhedra and estimation of oscillating integrals.} Funct. Anal. Appl. {\bf 10} (1976), 175-196. [5] {\sc Z\'u\~niga-Galindo W.A.}, \textit{Local zeta functions and Newton polyhedra.} Nagoya Math J. {\bf 172} (2003), 31-58. [6] {\sc Le\'on-Cardenal E. \& Z\'u\~niga-Galindo W.A.}, \textit{Zeta Functions for Non-degenerate Laurent Polynomials in Two Variables Over $p$-adic Fields.} Preprint, 2012.
%%%%%%%%%%%%511%%%%%%%%%%%
\subsection{\sffamily Funciones de Weil y m\'etricas {\footnotesize (RT, Pos)}} \label{reg-511} \index{bocardo gaspar miriam@Bocardo Gaspar Miriam!\ref{reg-511}}
\noindent {\bfseries Miriam Bocardo Gaspar}, {\tt miriam.bocardo@gmail.com}  {\slshape (Unidad Acad\'emica de Matem\'aticas (UAM))}\\
          \noindent Dada una variedad algebraica sobre un campo algebraicamente cerrado, definiremos los conceptos de divisor de Cartier y sus funciones de Weil asociadas. Demostraremos que existe una biyecci\'on entre las funciones de Weil asociadas a un divisor de Cartier D y las m\'etricas sobre el haz invertible asociado a D ${\cal O}_{X}(D)$.
%%%%%%%%%%%%522%%%%%%%%%%%
\subsection{\sffamily Estudio y desarrollo de problemas de Geometr\'\i a moderna y el uso de software din\'amico {\footnotesize (RT, 2Lic)}} \label{reg-522} \index{mendez gordillo alma rosa@M\'endez Gordillo Alma Rosa!\ref{reg-522}}
\noindent {\bfseries Alma Rosa M\'endez Gordillo}, {\tt almarosa9@gmail.com}  {\slshape (Facultad de Ciencias F\'\i sico Matem\'aticas  de la Universidad Michoacana de San Nicol\'as de Hidalgo (UMSNH))}\\
          \noindent El caso del Teorema de Feuerbach. El estudio de los contenidos de la Geometr\'\i a moderna es importante para los estudiantes de las Facultades de Matem\'aticas, ya que involucra nociones, conceptos y procedimientos que permiten profundizar en su estudio y resolver diversos tipos de problemas complejos; como es el caso de Teorema de Feuerbach, el cual asegura que en cualquier tri\'angulo, la Circunferencia de los nueve puntos es tangente a su circunferencia inscrita y cada una de sus circunferencias excritas, lo cual no resulta obvio. Quiz\'as, la incorporaci\'on del software din\'amico pueda ayudar a entender su enunciado y proporcione pistas para la demostraci\'on. Adem\'as, la funci\'on formativa de la geometr\'\i a ha sido esencial en el desarrollo personal de profesionales de las matem\'aticas y en general, de toda persona educada, pues presenta valores insustituibles que Thom (1973) resume en tres puntos: 1) La geometr\'\i a proporciona uno o m\'as puntos de vista en casi todas las \'areas de las matem\'aticas; 2) las interpretaciones geom\'etricas contin\'uan proporcionando visiones directoras del entendimiento intuitivo y avances en la mayor\'\i a de las \'areas de las matem\'aticas; y 3) las t\'ecnicas geom\'etricas proporcionan herramientas eficaces para resolver problemas en casi todas las \'areas de las Matem\'aticas. Al respecto, el modelo de razonamiento de los van Hiele (1986) indica que el razonamiento geom\'etrico de los estudiantes puede evolucionar desde las nociones m\'as intuitivas a otros niveles. Dicha teor\'\i a establece cinco Niveles de razonamiento. En el Nivel 1 el estudiante percibe los objetos como unidades, describe semejanzas y diferencias globales, pero no reconoce sus componentes y propiedades. En el Nivel 2 el estudiante percibe los objetos con sus partes y propiedades aunque no identifica las relaciones entre ellas; describe los objetos de manera informal pero no es capaz de hacer clasificaciones l\'ogicas; hace deducciones informales a partir de la experimentaci\'on. En el Nivel 3 el estudiante realiza clasificaciones l\'ogicas de los objetos, describe las figuras de manera formal, comprende los pasos individuales de un razonamiento l\'ogico, pero no es capaz de formalizar estos pasos, no comprende la estructura axiom\'atica de las matem\'aticas. En el Nivel 4 el estudiante es capaz de realizar razonamientos l\'ogicos formales, comprende la estructura axiom\'atica de las matem\'aticas y acepta la posibilidad de llegar a un mismo resultado desde distintas premisas. El modelo se\~nala la existencia de un Nivel 5, cuya caracter\'\i stica b\'asica es la capacidad para manejar, analizar y comparar diferentes geometr\'\i as; sin embargo, existe el reconocimiento que este nivel s\'olo se encuentra al alcance de algunos matem\'aticos profesionales y de ciertos estudiantes muy adelantados de las facultades de matem\'aticas. En esta tesis se estudian cuatro teoremas representativos de la Geometr\'\i a moderna: a) Teorema de la Circunferencia de los nueve puntos;  b) Teorema del eje radical de la circunferencia inscrita y excrita en un tri\'angulo; c) Teorema de Feuerbach; y d) Teorema de la Celda de Peaucellier. Aqu\'i se presenta el Teorema de Feuerbach, se desarrolla su demostraci\'on destacando las estrategias heur\'\i sticas utilizadas, y se comentan las ventajas de contar con una herramienta como el software din\'amico, que permite visualizar din\'amicamente los enunciados de los problemas y teoremas, as\'\i  como su contribuci\'on al entendimiento y soluci\'on de los problemas.
%%%%%%%%%%%%543%%%%%%%%%%%
\subsection{\sffamily C\'omo utilizar el algebra para descubrir la geometr\'\i a {\footnotesize (CPI, 1Lic)}} \label{reg-543} \index{gomez mont xavier@G�mez Mont Xavier!\ref{reg-543}}
\noindent {\bfseries Xavier  G\'omez Mont}, {\tt gmont@cimat.mx}  {\slshape (Centro de Investigaci\'on en Matem\'aticas (CIMAT))}\\
          \noindent Dada una funci\'on $f\colon\mathbb{R}^n\to\mathbb{R}$, queremos entender c\'omo las fibras de la funci\'on obtenidas al igualar su valor a una constante, f=c, va cambiando al mover la constante c (en el caso de $x^2+y^2+z^2$) ver\'\i amos esferas, que solo cambian el tipo topol\'ogico al pasar por el 0, pues pasa de vacio a esferas atraves de un punto. Las derivadas parciales de f juegan un papel sustantivo en entender este fen\'omeno, y describir\'e algunos m\'etodos que he desarrollado con mi alumnos para entender esta relaci\'on entre el algebra y la geometr\'\i a.
%%%%%%%%%%%%649%%%%%%%%%%%
\subsection{\sffamily Discriminantes y Maple {\footnotesize (CI, Pos)}} \label{reg-649} \index{kushner schnur alberto leon@Kushner Schnur Alberto Le\'on!\ref{reg-649}}
\noindent {\bfseries Alberto Le\'on Kushner Schnur}, {\tt kushner@servidor.unam.mx}  {\slshape (Facultad de Ciencias Universidad Nacional Aut\'onoma de M\'exico (UNAM))}\\
          \noindent En este trabajo se estudian los discriminantes de las c\'ubicas, cu\'articas, quinticas y s\'exticas. En los primeros dos casos, se resuelve el problema totalmente. En el caso de las quinticas y sexticas, se dan ejemplos del conjunto singular, incluyendo ejemplos num\'ericos.
%%%%%%%%%%%%1500%%%%%%%%%%%
\subsection{\sffamily C\'odigos detectores y correctores de errores {\footnotesize (CDV, 1Lic)}} \label{reg-1500} \index{maisner daniel bush@Maisner Daniel Bush!\ref{reg-1500}}
\noindent {\bfseries Daniel Bush Maisner}, {\tt maisner@gmail.com}  {\slshape (Universidad Aut\'onoma de la Ciudad de M\'exico (UACM))}\\
          \noindent En la pl\'atica se presentar\'a una introducci\'on a la teor\'\i a de c\'odigos finalizando con la presentaci\'on de algunas de las aplicaciones de la geometr\'\i a algebraica a esta \'area.
%%%%%%%%%%%%581%%%%%%%%%%%
\subsection{\sffamily Una nueva construcci\'on geom\'etrica de c\'odigos algebraico geom\'etricos {\footnotesize (RI, Pos)}} \label{reg-581} \index{de la rosa navarro brenda leticia@De La Rosa Navarro Brenda Leticia!\ref{reg-581}}
\noindent {\bfseries Brenda Leticia De La Rosa Navarro}, {\tt brenda@ifm.umich.mx}  {\slshape (Instituto de F\'\i sica y Matem\'aticas (IFM) de la Universidad Michoacana de San Nicol\'as de Hidalgo (UMSNH))}\\
          \noindent El presente trabajo de investigaci\'on consiste en aplicar las t\'ecnicas de la geometr\'\i a algebraica a la construcci\'on de c\'odigos algebraico geom\'etricos con buenos par\'ametros, mediante la utilizaci\'on de las geometr\'\i as de superficies racionales proyectivas lisas, que mejoren los ya existentes. Como los c\'odigos algebraico geom\'etricos constituyen una subfamilia de c\'odigos lineales, dar\'e algunas nociones necesarias para el estudio de estos.
%%%%%%%%%%%%910%%%%%%%%%%%
\subsection{\sffamily Clasificaci\'on de los subesquemas cerrados de esquemas v\'\i a el concepto de las gavillas casi coherentes {\footnotesize (RT, Pos)}} \label{reg-910} \index{fr\i as medina juan bosco@Fr\'\i as Medina Juan Bosco!\ref{reg-910}}
\noindent {\bfseries Juan Bosco Fr\'\i as Medina}, {\tt boscof@ifm.umich.mx}  {\slshape (Posgrado Conjunto en Ciencias Matem\'aticas UNAM-UMSNH (PCCM UNAM-UMSNH))}\\
          \noindent El objetivo de esta pl\'atica es realizar la clasificaci\'on de los subesquemas cerrados de un esquema, para ello, utilizaremos algunos resultados sobre las gavillas casi coherentes.
%%%%%%%%%%%%1072%%%%%%%%%%%
\subsection{\sffamily El Anillo de Cox de las superficies proyectivas racionales {\footnotesize (CI, Inv)}} \label{reg-1072} \index{lahyane mustapha@Lahyane Mustapha!\ref{reg-1072}}
\noindent {\bfseries Mustapha  Lahyane}, {\tt lahyane@ifm.umich.mx}  {\slshape (Instituto de F\'\i sica y Matem\'aticas (IFM) Universidad Michoacana de San Nicol\'as de Hidalgo (UMSNH))}\\
          \noindent El objetivo principal del seminario es de caracterizar las superficies proyectivas racionales algebraicamente y geom\'etricamente. El campo base de nuestras superficies es algebraicamente cerrado de cualquiera caracter\'\i stica.
%%%%%%%%%%%%408%%%%%%%%%%%
\subsection{\sffamily De la conjugaci\'on de matrices a la construcci\'on de cocientes algebraicos {\footnotesize (CPI, 2Lic)}} \label{reg-408} \index{reynoso alcantara claudia@Reynoso Alc\'antara Claudia!\ref{reg-408}}
\noindent {\bfseries Claudia  Reynoso Alc\'antara}, {\tt claudiagto@gmail.com}  {\slshape (Departamento de Matem\'aticas, Universidad de Guanajuato)}\\
          \noindent Una de las principales tareas de la matem\'atica es clasificar objetos; una manera de hacerlo es construyendo espacios que parametricen los  objetos que se desea clasificar salvo relaciones de equivalencia. El objetivo de esta pl\'atica es, a trav\'es de la conjugaci\'on de matrices,  introducir al oyente a la Teor\'\i a Geom\'etrica de Invariantes (en ingl\'es  GIT). El resultado principal de esta teor\'\i a nos dice que, bajo ciertas  condiciones, es posible construir espacios con buenas propiedades  algebraicas que parametrizan objetos propios de la geometr\'\i a algebraica.
%%%%%%%%%%%%644%%%%%%%%%%%
\subsection{\sffamily Ciclos algebraicos sobre variedades abelianas de dimensi\'on 4 {\footnotesize (CI, Pos)}} \label{reg-644} \index{qui\~nones estrella russell aaron@Qui\~nones Estrella Russell Aar\'on!\ref{reg-644}}
\noindent {\bfseries Russell Aar\'on Qui\~nones Estrella}, {\tt rusell.quinones@unach.mx}  {\slshape (Universidad Aut\'onoma de Chiapas (UNACH))}\\
          \noindent En la pl\'atica se dar\'a la construcci\'on de elemento no trivial en el grupo de Griffiths superior $Griff^{3,2}(A^4)$, donde $A^4$ representa una variedad abeliana compleja gen\'erica de dimensi\'on 4. La idea esencial es utilizar el hecho que $A^4$ puede ser realizada como una variedad de Prym generalizada la cual contiene de manera natural algunas curvas, i.e. ciclos de dimensi\'on 1.
%%%%%%%%%%%%450%%%%%%%%%%%
\subsection{\sffamily De las curvas a las superficies {\footnotesize (CPI, 2Lic)}} \label{reg-450} \index{garc\i a zamora alexis miguel@Garc\'\i a Zamora Alexis Miguel!\ref{reg-450}}
\noindent {\bfseries Alexis Miguel Garc\'\i a Zamora}, {\tt alexiszamora06@gmail.com}  {\slshape (Universidad Aut\'onoma de Zacatecas (UAZ))}\\
          \noindent En esta pl\'atica panor\'amica explicaremos la clasificai\'on decurvas algebraicas de acuerdo al comportamiento del divisor can\'onico. Luego veremos la generalizaci\'on de este procedimiento para el caso de superficies y cu\'ales son las dificultades que aparecen en este caso.
