\section{\'Algebra}

%%%%%%%%%%%%1075%%%%%%%%%%%
\subsection{\sffamily La versi\'on categ\'orica del \'algebra universal {\footnotesize (CDV, 2Lic)}} \label{reg-1075} \index{marmolejo francisco@Marmolejo Francisco!\ref{reg-1075}}
\noindent {\bfseries Francisco  Marmolejo}, {\tt quico@matem.unam.mx}  {\slshape (Instituto de Matem\'aticas  Universidad Nacional Aut\'onoma de M\'exico (UNAM))}\\
\noindent La versi\'on categ\'orica del \'algebra universal est\'a basado en el concepto de m\'onada. En esta pl\'atica comenzaremos con la definici\'on de m\'onada, y su relaci\'on con funtores adjuntos, para posteriormente analizar la relaci\'on de \'estas con el \'algebra universal. Posteriormente veremos morfismos de m\'onadas y leyes distributivas entre m\'onadas y analizaremos varios ejemplos.
%%%%%%%%%%%%1168%%%%%%%%%%%
\subsection{\sffamily Otra caracterizaci\'on de los grupos c\'{\i}clicos finitos {\footnotesize (CDV, 2Lic)}} \label{reg-1168} \index{morales rodriguez juan@Morales Rodr\'iguez Juan!\ref{reg-1168}}
\noindent {\bfseries Juan  Morales Rodr\'iguez}, {\tt juanmoralesrodriguez@gmail.com}  {\slshape (Facultad de Ciencias, Universidad Nacional Aut\'onoma de M\'exico (UNAM))}\\
          \noindent En cada grupo c\'iclico finito, existe un elemento g de orden m\'aximo, necesariamente en su centro, tal que el subgrupo generado por cada elemento diferente del id\'entico tiene una intersecci\'on no trivial con el subgrupo generado por g. Demostraremos que la propiedad anterior caracteriza a los grupos c\'iclicos finitos y la usaremos para probar dos conocidos resultados, primero, que en un grupo abeliano finito G, si g  es un elemento de orden m\'aximo, G es producto directo del subgrupo generado por g con un subgrupo M, y segundo, que un grupo finito es c\'iclico si y s\'olo si es abeliano y por cada primo p que divide a su orden, el grupo tiene s\'olo un subgrupo de orden p.
%%%%%%%%%%%%247%%%%%%%%%%%
\subsection{\sffamily On the union of increasing chains of torsion-free modules {\footnotesize (RI, Pos)}} \label{reg-247} \index{macias diaz jorge eduardo@Mac\'ias D\'iaz Jorge Eduardo!\ref{reg-247}}
\noindent {\bfseries Jorge Eduardo Mac\'ias D\'iaz}, {\tt siegs\_wehrmacht@hotmail.com}  {\slshape (Universidad Aut\'onoma de Aguascalientes, Departamento de Matem\'aticas y F\'isica)}\\
          \noindent Motivated by Hill's criterion of freeness for abelian groups, we establish a generalization of that result to categories $\mathcal {C}$ of torsion-free modules over integral domains, which are closed with respect to the formation of direct sums, and in which every object can be decomposed into direct sums of objects of $\mathcal {C}$ of rank at most a fixed limit cardinal number $\kappa$. Our main result states that a module belongs to $\mathcal {C}$ if it is the union of a continuous, well-ordered, ascending chain of length $\kappa$, consisting of pure submodules which are objects of $\mathcal {C}$. As corollaries, we derive versions of Hill's theorem for some classes of torsion-free modules over domains, and a generalization of a well-known result by Kaplansky.
%%%%%%%%%%%%410%%%%%%%%%%%
\subsection{\sffamily Combinatoria y representaciones del grupo sim\'etrico {\footnotesize (CDV, 2Lic)}} \label{reg-410} \index{vallejo ruiz ernesto@Vallejo Ruiz Ernesto!\ref{reg-410}}
\noindent {\bfseries Ernesto  Vallejo Ruiz}, {\tt ernevallejo@gmail.com}  {\slshape (Universidad Nacional Aut\'onoma de M\'exico (UNAM), Centro de Ciencias Matem\'aticas, Morelia.)}\\
          \noindent La teor\'ia de representaciones de grupos se encarga de estudiar ciertos tipos de simetr\'ias en espacios vectoriales. En esta pl\'atica, dirigida a estudiantes de licenciatura, introducimos las nociones b\'asicas de teor\'ia de representaciones y las ejemplificamos con el grupo sim\'etrico. En este caso la teor\'ia tiene un fuerte sabor combinatorio y los resultados resultan elegantes y muy bellos.
%%%%%%%%%%%%591%%%%%%%%%%%
%\subsection{\sffamily La conjetura finitista: Una conjetura Homol\'ogica {\footnotesize (CDV, Pos)}} \label{reg-591} \index{mendoza hernandez octavio@Mendoza Hern\'andez Octavio!\ref{reg-591}}
%\noindent {\bfseries Octavio  Mendoza Hern\'andez}, {\tt omendoza@matem.unam.mx}  {\slshape (Instituto de Matem\'aticas, Universidad Nacional Aut\'onoma de M\'exico (UNAM))}\\
 %         \noindent A finales de los a\~nos cincuenta, Rosenberg y Zelinsky, conjeturaron que la dimensi\'on proyectiva, de los m\'odulos (sobre una algebra de dimensi\'on finita) finitamente generados (de dimensi\'on proyectiva finita), es uniformemente acotada. Dicha conjetura se le conoce como ``la conjetura finitista'', y est\'a conectada con otras importantes conjeturas homol\'ogicas. La conjetura finitista sigue abierta hasta la fecha. Adem\'as, ha servido como fuente de inspiraci\'on y motivaci\'on; tanto para pensar (desde otros puntos de vista) cuestiones homol\'ogicas, como para impulsar el desarrollo de nuevas t\'ecnicas homol\'ogicas. En  esta charla introducir\'e en detalle el planteamiento de dicha conjetura, su trascendencia e importancia y as\'i como tambi\'en algunos avances obtenidos en los \'ultimos tiempos.
%%%%%%%%%%%%400%%%%%%%%%%%
\subsection{\sffamily Polinomios c\'ubicos de permutaci\'on autoinvertibles sobre $\mathbb{Z}_{p^n}$ con $p>7$ primo {\footnotesize (RI, 2Lic)}} \label{reg-400} \index{rubio barrios carlos jacob@Rubio Barrios Carlos Jacob!\ref{reg-400}}
\noindent {\bfseries Carlos Jacob Rubio Barrios}, {\tt carlos.rubio@uady.mx}  {\slshape (Facultad de Matem\'aticas Universidad Aut\'onoma de Yucat\'an (UADY))}\\
          \noindent En esta pl\'atica daremos condiciones necesarias y suficientes para que un polinomio de permutaci\'on c\'ubico sea autoinvertible en el anillo $\mathbb{Z}_{p^n}$ con $p>7$ n\'umero primo y $n>1$ entero.
%%%%%%%%%%%%1224%%%%%%%%%%%
\subsection{\sffamily Clases de M\'odulos, la visi\'on de Francisco Raggi {\footnotesize (CPI, Pos)}} \label{reg-1224} \index{signoret carlos jose@Signoret Carlos Jos\'e!\ref{reg-1224}}
\noindent {\bfseries Carlos Jos\'e Signoret}, {\tt casi@xanum.uam.mx}  {\slshape (Departamento de Matem\'aticas, Universidad Aut\'onoma Metropolitana Iztapalapa (UAM-I))}\\
          \noindent En esta pl\'atica presentaremos un panorama de los resultados obtenidos por Francisco Raggi en colaboraci\'on con el expositor alrededor de temas como: teor\'ia de la dimensi\'on, subcategor\'ias de Serre, clases de m\'odulos y filtros lineales, entre otros. Trataremos de comunicar la forma en que Francisco ve\'ia los distintos objetos aqu\'i mencionados, as\'i como su particular visi\'on global del \'algebra.
%%%%%%%%%%%%490%%%%%%%%%%%
\subsection{\sffamily Ret\'iculas de Prerradicales {\footnotesize (CDV, 2Lic)}} \label{reg-490} \index{fernandez-alonso gonzalez rogelio@Fern\'andez-Alonso Gonz\'alez Rogelio!\ref{reg-490}}
\noindent {\bfseries Rogelio  Fern\'andez-Alonso Gonz\'alez}, {\tt rojo99@prodigy.net.mx}  {\slshape (Universidad Aut\'onoma Metropolitana (UAM) Iztapalapa Departamento de Matem\'aticas)}\\
          \noindent As\'i como sucede con las categor\'ias de m\'odulos, las ret\'iculas de prerradicales reflejan las caracter\'isticas del anillo correspondiente. En esta charla se plantear\'an las propiedades generales de la ret\'icula de prerradicales asociada a un anillo asociativo con uno. Tambi\'en se describir\'an algunos ejemplos de ret\'iculas de prerradicales correspondientes a anillos espec\'ificos.
%%%%%%%%%%%%1460%%%%%%%%%%%
\subsection{\sffamily Algunos aspectos reticulares del conjunto de derivadas en el marco R-tors {\footnotesize (RI, Inv)}} \label{reg-1460} \index{zaldivar corichi luis angel@Zald\'ivar Corichi Luis \'Angel!\ref{reg-1460}}
\noindent {\bfseries Luis \'Angel Zald\'ivar Corichi}, {\tt angelus31415@gmail.com}  {\slshape (Instituto de Matem\'aticas, CU (IMATE))}\\
          \noindent Dado un marco $A$, una derivada en $A$ es una funci\'on $d:A\rightarrow A$ tal que satisface:\begin{itemize}\item[1]$a\leq d(a)$ para todo $a\in A$.\item[2]Si $a\leq b$ en $A$ entonces $d(a)\leq d(b)$.\end{itemize}El conjunto de todas estas funciones $D(A)$ tiene una estructura de ret\'icula completa. En este reporte de investigaci\'on examinaremos algunas de las propiedades que cumple $D(A)$ poniendo particular atenci\'on cuando $A=R-tors$ el marco de todas las teor\'ias de torsi\'on hereditarias en la categor\'ia de m\'odulos izquierdos sobre un anillo asociativo con uno $R$.
%%%%%%%%%%%%1282%%%%%%%%%%%
\subsection{\sffamily Measuring modules: alternative perspectives in module theory {\footnotesize (CI, Inv)}} \label{reg-1282} \index{lopez permouth sergio roberto@L\'opez Permouth Sergio Roberto!\ref{reg-1282}}
\noindent {\bfseries Sergio Roberto L\'opez Permouth}, {\tt lopez@ohio.edu}  {\slshape (Ohio University (OU))}\\
          \noindent We will consider various new ways to gauge the projectivity or injectivity of modules. As an illustration of the usefulness of these new approaches and in contrast with the traditional approach when research has focused on modules which are ``as projective (or injective) as possible'', we will focus on modules which are weakest in terms of projectivity or injectivity. We will show how all of these related notions are interesting in their own right.
%%%%%%%%%%%%345%%%%%%%%%%%
\subsection{\sffamily Dimensi\'on de Krull y Dimensi\'on Cl\'asica de Krull de M\'odulos {\footnotesize (CI, Inv)}} \label{reg-345} \index{castro perez jaime@Castro P\'erez Jaime!\ref{reg-345}}
\noindent {\bfseries Jaime  Castro P\'erez}, {\tt jcastrop@itesm.mx}  {\slshape (ITESM CCM Departamento de F\'isica y Matem\'aticas)}\\
\noindent {\it  Coautor: Jos\'e  R\'ios Montes        }\\
\noindent  Para un $R$-m\'odulo izquierdo $M$ definimos el concepto de Dimensi\'{o}n Cl\'{a}sica de Krull relativa a una teor\'{\i}a de torsi\'{o}n hereditaria $\tau \in M$-tors (denotada como $cl.K_{\tau }\dim \left( M\right) )$. Probamos que si $M$ es generador proyectivo de la categor\'{\i}a $\sigma  \left[ M\right] $ y $\tau \in M$-tors, tal que $M$ tiene dimensi\'{o}n $\tau $-Krull, en el sentido de Jategaonkar (denotada como $k_{\tau }\left( M\right) $), entonces $cl.K_{\tau }\dim \left( M\right) \leq k_{\tau }\left( M\right)$. Tambi\'{e}n probamos que si $M$ es noetheriano $\tau $ -completamente acotado, generador proyectivo de \ la categor\'{\i}a $\sigma  \left[ M\right] $ y $\tau \in M$-tors, tal que $M$ es libre de $\tau $-torsi\'{o}n, entonces $cl.K_{\tau }\dim \left( M\right) =k_{\tau }\left( M\right)  $. Estos resultados generalizan los resultados obtenidos por Peter L. Vachuska y Krause respectivamente.
%%%%%%%%%%%%283%%%%%%%%%%%
\subsection{\sffamily Anillos para los  cuales la ret\'icula de teor\'ias de torsi\'on hereditarias y la ret\'icula de clases naturales son isomorfas {\footnotesize (RI, Inv)}} \label{reg-283} \index{vilchis montalvo ivan fernando@Vilchis Montalvo Iv\'an Fernando!\ref{reg-283}}
\noindent {\bfseries Iv\'an Fernando Vilchis Montalvo}, {\tt vilchis.f@gmail.com}  {\slshape (Departamento de Matem\'aticas, Facultad de Ciencias, Universidad Nacional Aut\'onoma de M\'exico (UNAM))}\\
          \noindent En este trabajo damos una funci\'on suprayectiva de la ret\'iculas de clases naturales a la ret\'icula de clases de torsi\'on hereditarias que preserva orden e \'infimos. Demostramos que es un morfismo de ret\'iculas precisamente cuando es un isomorfismo de ret\'iculas y esto pasa si y s\'olo si R es un anillo semiartiniano. Tambi\'en estudiamos las fibras de la funci\'on.
%%%%%%%%%%%%285%%%%%%%%%%%
\subsection{\sffamily Acerca de anillos artinianos de ideales principales. {\footnotesize (RI, Inv)}} \label{reg-285} \index{cejudo castilla cesar@Cejudo Castilla C\'esar!\ref{reg-285}}
\noindent {\bfseries C\'esar Cejudo Castilla}, {\tt cesarcc@ciencias.unam.mx}  {\slshape (Departamento de Matem\'aticas, Facultad de Ciencias de la Universidad Nacional Aut\'onoma de M\'exico (UNAM))}\\
          \noindent En este trabajo obtenemos algunas caracterizaciones de anillos artinianos de ideales principales mediante el uso de propiedades de grandes ret\'iculas de clases de m\'odulos.
%%%%%%%%%%%%346%%%%%%%%%%%
\subsection{\sffamily Cuando Sherlock Holmes usa Calvin Klein: deduciendo grupos por sus marcas {\footnotesize (CDV, 2Lic)}} \label{reg-346} \index{valero elizondo luis@Valero Elizondo Luis!\ref{reg-346}}
\noindent {\bfseries Luis  Valero Elizondo}, {\tt valero@fismat.umich.mx}  {\slshape (Universidad Michoacana de San Nicol\'as de Hidalgo (UMSNH))}\\
          \noindent Las tablas de marcas tienen gran informaci\'on sobre el grupo al cu\'al describen. En esta pl\'atica definiremos la matriz de marcas, y veremos c\'omo puede usarse para determinar algunos grupos hasta isomorfismo. Tambi\'en veremos un ejemplo de grupos no isomorfos con tablas de marcas isomorfas.
%%%%%%%%%%%%707%%%%%%%%%%%
\subsection{\sffamily La funci\'on zeta del anillo de Burnside del grupo alternante $A_4$ {\footnotesize (RI, Pos)}} \label{reg-707} \index{villa hernandez david@Villa Hern\'andez David!\ref{reg-707}}
\noindent {\bfseries David  Villa Hern\'andez}, {\tt dvilla@fcfm.buap.mx}  {\slshape (Facultad de Cs. F\'isico Matem\'aticas, Benem\'erita Universidad Aut\'onoma de Puebla (BUAP))}\\
          \noindent  En base a la descomposici\'on del anillo de Burnside en sus componentes solubles, las tablas de marcas y resultados anteriores, obtenidos para las funciones zeta del anillo de Burnside para grupos c\'iclicos de orden una potencia de un primo racional, realizaremos el c\'alculo de las funciones zeta para el anillo de Burnside del grupo alternante $A_4$ en los casos local y global.
%%%%%%%%%%%%510%%%%%%%%%%%
\subsection{\sffamily Una generalizaci\'on a la categor\'ia de biconjuntos {\footnotesize (CPI, Pos)}} \label{reg-510} \index{tacho jesus tadeo ibarra@Tacho Jes\'us Tadeo Ibarra!\ref{reg-510}}
\noindent {\bfseries Jes\'us Tadeo Ibarra Tacho}, {\tt tadeo@matmor.unam.mx}  {\slshape (Centro de Ciencias Matem\'aticas de la Universidad Nacional Aut\'onoma de M\'exico (UNAM), Campus Morelia)}\\
          \noindent En la presente pl\'atica definiremos una categor\'ia  aditiva de tal forma que contiene una subcategor\'ia plena, equivalente a la categor\'ia de biconjuntos definida por Serge Bouc. Probaremos que cada objeto en esta categor\'ia se escribe de manera \'unica salvo isomorfismo como suma directa de objetos en la categor\'ia de biconjuntos de tal forma que las correspondientes categor\'ias de funtores aditivos a grupos abelianos resultan ser isomorfas. Tambi\'en hablaremos sobre funtores de biconjuntos cl\'asicos definidos en esta categor\'ia.
%%%%%%%%%%%%1684%%%%%%%%%%%
\subsection{\sffamily Teor\'ia de inclinaci\'on en categor\'ias de funtores {\footnotesize (RT, Pos)}} \label{reg-1684} \index{ortiz morales martin@Ort\'iz Morales Mart\'in!\ref{reg-1684}}
\noindent {\bfseries Mart\'in Ort\'iz Morales}, {\tt mortiz@matmor.unam.mx}  {\slshape (Tecnol\'ogico de Estudios Superiores de Jocotitlan (TESJO))}\\
          \noindent En esta pl\'atica se hablar\'a de la generalizaci\'on de la teor\'ia de inclinaci\'on en la categor\'ia de m\'odulos  $\mathrm{mod}(\Lambda)$, con $\Lambda$ una $K$-\'algebra de dimensi\'on finita, a la categor\'ia de funtores finitamente  presentados $(\mathrm{mod}(\mathcal{C}))$  que van de $\mathcal{C}$ a  la categor\'ia de grupos abelianos $\mathrm{Ab}$,  donde $\mathcal{C}$ es una categor\'ia aditiva esquel\'eticamente peque\~na donde los idempotentes se dividen.    Adem\'as se mostrar\'a que para  el \'algebra de carcaj $K(Q)$, con $Q$ un carcaj infinito localmente finito sin caminos de longitud infinita,  las secciones sin caminos de longitud infinita en la componente preproyectiva  forman una categor\'ia de inclinaci\'on, teniendo resultados an\'alogos a los expuestos en [2]. Tambi\'en se mostrar\'a una generalizaci\'on de los resultados expuestos en [1].\\ 1. E. CLINE, B. PARSHALL and L. SCOTT, \emph{Derived categories and Morita theory}. Algebra 104 (1986) 397-409.\\ 2. D. HAPPEL and C. M. RINGEL, \emph{Tilted algebras}, Trans. Amer. Math. Soc. 21A (1982) 339-443.
%%%%%%%%%%%%423%%%%%%%%%%%
\subsection{\sffamily Aplicaciones de la forma normal de Smith de una matr\'iz entera {\footnotesize (CDV, 2Lic)}} \label{reg-423} \index{villarreal rodriguez rafael heraclio@Villarreal Rodr\'iguez Rafael Heraclio!\ref{reg-423}}
\noindent {\bfseries Rafael Heraclio Villarreal Rodr\'iguez}, {\tt vila@math.cinvestav.mx}  {\slshape (Centro de Investigaci\'on y de Estudios Avanzados del IPN (Cinvestav-IPN))}\\
          \noindent La forma normal de Smith de una matriz entera es central para determinar las soluciones enteras de ecuaciones lineales con coeficientes enteros. En esta pl\'atica presentaremos otras aplicaciones. En particular, veremos c\'omo se determina el subgrupo de torsi\'on y la parte libre de un grupo abeliano finitamente generado. La torsi\'on es muy \'util para determinar el grado de ciertas variedades proyectivas  sobre un campo finito que aparecen en teor\'ia algebraica de c\'odigos.
%%%%%%%%%%%%321%%%%%%%%%%%
\subsection{\sffamily Series formales sobre gr\'aficas orientadas finitas {\footnotesize (CDV, 2Lic)}} \label{reg-321} \index{bautista ramos raymundo@Bautista Ramos Raymundo!\ref{reg-321}}
\noindent {\bfseries Raymundo  Bautista Ramos}, {\tt raymundo@matmor.unam.mx}  {\slshape (Centro de Ciencias Matem\'aticas (CCM) de la Universidad Nacional Aut\'onoma de M\'exico (UNAM))}\\
          \noindent Las series formales sobre una variable son una extensi\'on natural de los polinomios sobre una variable y tienen muchas aplicaciones en combinatoria y en la teor\'ia de \'algebras v\'ertice. Si G es una gr\'afica finita y k es un campo se tiene el \'algebra de caminos C(G,k) que es la dada por combinaciones lineales sobre k de los caminos orientados de G. Esta es una idea similar a la de los polinomios en una variable. En la pl\'atica introducimos el \'algebra F(G,k) de series formales de G sobre k. \'Esta es similar a las series formales en una variable sobre k. Veremos algunas de sus aplicaciones al estudio de las llamadas \'Algebras con Potencial.
%%%%%%%%%%%%837%%%%%%%%%%%
\subsection{\sffamily Operadores V\'ertice y \'Algebras de Lie Afines {\footnotesize (CPI, 2Lic)}} \label{reg-837} \index{espinoza arce jose angel@Espinoza Arce Jos\'e \'Angel!\ref{reg-837}}
\noindent {\bfseries Jos\'e \'Angel Espinoza Arce}, {\tt angel@matmor.unam.mx}  {\slshape (Centro de Ciencias Matem\'aticas, UNAM, Campus Morelia)}\\
          \noindent Las \'algebras de Lie afines son algunos de los primeros ejemplos, despu\'es de las \'algebras de Lie cl\'asicas, de \'algebras Kac-Moody. En pocas palabras, dichas \'algebras son extensiones centrales del \'algebra de Lie de alg\'un grupo de lazos; es decir, son \'algebras del tipo\begin{equation}\nonumber \widehat{\mathfrak{g}}=\mathfrak{g}\otimes \mathbb{C}[t,t^{-1}]\oplus \mathbb{C}\,c,\end{equation}donde $\mathfrak{g}$ es un \'algebra de Lie cl\'asica, $c$ es un elemento central y $[a\otimes t^m,b\otimes t^n]=[a,b]\otimes t^{m+n} + m \langle a,b\rangle \delta_{m+n,0}\,c$, para todos $a,b\in \mathfrak{g}$ y $m,n\in\mathbb{Z}$ ($\langle\,,\,\rangle$ denota la forma de Killing sobre $\mathfrak{g}$). A principios de los 80's fue introducida un \emph{representaci\'on b\'asica} $V$ (Frenkel-Kac, Segal) para dichas \'algebras utilizando unos objetos llamados ``Operadores V\'ertice'', tal construcci\'on utiliza \'unicamente la informaci\'on contenida en la matriz de Cartan de $\mathfrak{g}$. En esta charla s\'olo consideraremos el caso  $\mathfrak{g}=\mathfrak{sl}_2$. En tal caso, $V$ se descompone en dos subm\'odulos irreducibles de nivel $1$ (el elemento central act\'ua como una constante por la identidad, tal constante es llamada el nivel de la representaci\'on), generados por \emph{vectores de peso m\'aximal} $v_0,v_1\in V$:\begin{equation}\nonumber V=U(\widehat{\mathfrak{sl}}_2)\cdot v_0\; \oplus\; U(\widehat{\mathfrak{sl}}_2)\cdot v_1.\end{equation}Es posible encontrar representaciones irreducibles de nivel m\'as alto (entero) dentro de $V$ mismo, considerando para $k>1$ la \emph{sub\'algebra completa de profundidad $k$}\begin{equation}\nonumber \widehat{\mathfrak{g}}_{[k]}=\mathfrak{g}\otimes \mathbb{C}[t^k,t^{-k}]\oplus \mathbb{C}\,c\subset\widehat{\mathfrak{g}}.\end{equation}El punto es que $\widehat{\mathfrak{g}}\cong \widehat{\mathfrak{g}}_{[k]}$, v\'ia la asignaci\'on $a\otimes t^m\mapsto a\otimes t^{mk}$ para todo $a\in\mathfrak{g}$, $c\mapsto k\,c$, y, claramente todo $\widehat{\mathfrak{g}}$-m\'odulo de nivel $l$ es naturalmente un $\widehat{\mathfrak{g}}$-m\'odulo de nivel $lk$ atrav\'es de esta identificaci\'on. Actualmente no se conoce una descomposici\'on de $V$ en subm\'odulos irreducibles de nivel $k$. Se sabe que existen $k+1$ vectores de peso maximal $v_0,v_1,\ldots,v_k\in V$ de tal forma que \begin{equation}\nonumber U(\widehat{\mathfrak{sl}_2}_{[k]})\cdot v_0\; \oplus\; U(\widehat{\mathfrak{sl}_2}_{[k]})\cdot v_1\;\oplus \cdots\oplus \;U(\widehat{\mathfrak{sl}_2}_{[k]})\cdot v_k\subsetneq V,\end{equation}tal lista de m\'odulos irreducibles es la lista completa de clases de equivalencia de representaciones irreducibles de nivel $k$, sin embargo no se sabe con que multiplicidad aparece cada uno de estos m\'odulos. En la charla mencionar\'e algunos avances en esta direcci\'on.
%%%%%%%%%%%%965%%%%%%%%%%%
\subsection{\sffamily Sobre el orden de polinomios de permutaci\'on {\footnotesize (RI, 2Lic)}} \label{reg-965} \index{sozaya chan jose antonio@Sozaya Chan Jos\'e Antonio!\ref{reg-965}}
\noindent {\bfseries Jos\'e Antonio Sozaya Chan}, {\tt soca\_8817@hotmail.com}  {\slshape (Universidad Aut\'onoma de Yucat\'an (UADY))}\\
\noindent {\it  Coautores: Javier  D\'iaz Vargas, Horacio  Tapia Recillas, Carlos  Rubio Barrios    }\\
\noindent Dado un anillo conmutativo con identidad $R$ finito, se dice que un elemento $f\in R[x]$  es un \textit{polinomio de permutaci\'on} sobre R si $f$ act\'ua como una permutaci\'on sobre $R$, i.e. si el mapeo $a\mapsto f(a)$ es una biyecci\'on. Los polinomios de permutaci\'on han sido ampliamente estudiados por sus aplicaciones en Criptograf\'ia y Teor\'ia de C\'odigos, sin embargo la mayor\'ia de los resultados surgen bajo la suposici\'on de que $R$ denota un campo finito. Un resultado importante acerca de los polinomios de permutaci\'on sobre anillos de enteros m\'odulo potencia de primo aparece en la literatura se enuncia como sigue:\\ {\bf Teorema 1:} Un polinomio $f \in (\mathbb{Z}/p^{\alpha}\mathbb{Z})[x]$ con $p$ primo y $\alpha>1$ permuta $\mathbb{Z}/p^{\alpha}\mathbb{Z}$ si y s\'olamente si $\pi(f)\in(\mathbb{Z}/p\mathbb{Z})[x]$ es un polinomio que permuta $\mathbb{Z}/p\mathbb{Z}$ cuya derivada no tiene ra\'ices en $\mathbb{Z}/p\mathbb{Z}$.\\ Asociado a los polinomios de permutaci\'on, un concepto de especial inter\'es por cuestiones pr\'acticas es el de \textit{orden}. Siendo $f\in R[x]$ un polinomio de permutaci\'on sobre $R$, se define el orden de $f$ denotado por $\mathrm{ord}(f)$ como el m\'inimo entero positivo $k$ tal que la $k$-\'esima composici\'on de $f$ consigo mismo induce la funci\'on identidad sobre $R$. El orden de todo polinomio de permutaci\'on siempre es finito y divide a $n!$ donde $n=|R|$.\\ {\bf Teorema 2:} Sea $f(x)=ax+x^{2}g(x)\in(\mathbb{Z}/p^{\alpha}\mathbb{Z})[x]$ un polinomio de permutaci\'on. Si se supone que $\mathrm{ord}(f)$ es primo y $a\not\equiv 1\mod(p)$ entonces $\mathrm{ord}(f)$ coincide con el orden de $\pi(a)$ en el grupo de unidades de $\mathbb{Z}/p\mathbb{Z}$, donde $\pi:\mathbb{Z}/p^{\alpha}\mathbb{Z}\to \mathbb{Z}/p\mathbb{Z}$ denota el epimorfismo can\'onico.\\ En t\'erminos generales, determinar si un polinomio dado induce una permutaci\'on as\'i como encontrar su orden es un problema no trivial, no obstante considerando polinomios de forma espec\'ifica se establece el siguiente teorema:\\ {\bf Teorema 3:} Sea $ax+bx^{k}\in(\mathbb{Z}/m\mathbb{Z})[x]$ un binomio de permutaci\'on de orden $s\leq\mathrm{log}_{k}(p)$ para cada divisor primo $p$ de $m$, entonces $b$ es nilpotente, $k\not\equiv 1\mod(s)$ y $a^{s}=1$.\\ Las condiciones necesarias y suficientes para que el binomio $ax+bx^{2}$ con coeficientes en el anillo $\mathbb{Z}/p^{\alpha}\mathbb{Z}$ sea un polinomio de permutaci\'on de orden $s\in\{2,3,5,7\}$ con $s\leq \mathrm{log}_{2}(p)$ quedan completamente caracterizadas y son relativamente simples, siendo: $1+a+\cdots+a^{s-1}=b^{s}=0$. En adici\'on, ning\'un binomio de permutaci\'on de grado impar y libre de t\'ermino constante puede ser su propio inverso (i.e., de orden $1$ o $2$) sobre estos anillos para $p$ lo suficientemente grande.
%%%%%%%%%%%%375%%%%%%%%%%%
\subsection{\sffamily \'Algebra Conmutativa y Teor\'\i a de C\'odigos {\footnotesize (CDV, 2Lic)}} \label{reg-375} \index{tapia-recillas horacio@Tapia-Recillas Horacio!\ref{reg-375}}
\noindent {\bfseries Horacio  Tapia-Recillas}, {\tt hrt@xanum.uam.mx}  {\slshape (Departamento de Matem\'aticas,  Universidad Aut\'onoma Metropolitana- Iztapalapa (UAM-I))}\\
          \noindent Hasta hace poco tiempo, \'areas de la Matem\'atica como el \'Algebra Conmutativa, Geometr\'\i a Algebraica y Teor\'\i a de N\'umeros, entre otras, se consideraban lejos de tener una aplicaci\'on en la soluci\'on de problemas pr\'acticos y vinculados con la vida cotidiana. Uno de estos problemas est\'a relacionado con la trasmisi\'on, almacenamiento y seguridad de la informaci\'on. En esta pl\'atica se motivar\'a el estudio de los C\'odigos Lineales Detectores-Correctores de Errores y se mencionar\'an algunas aplicaciones actuales relevantes en la vida diaria.  Se dar\'an algunos ejemplos de c\'odigos los cuales motivan el uso de conceptos y resultados de \'Algebra Conmutativa en la Teor\'\i a de C\'odigos Lineales. Los requisitos para seguir la pl\'atica son m\'\i nimos: conceptos b\'asicos de \'Algebra.

%%%%%%%%%%%%1243%%%%%%%%%%%
\subsection{\sffamily \'Algebras y super\'algebras de Lie ?`C\'omo se clasifican? {\footnotesize (CDV, 2Lic)}} \label{reg-1243} \index{salgado gil@Salgado Gil!\ref{reg-1243}}
\noindent {\bfseries Gil  Salgado}, {\tt gil.salgado@gmail.com}  {\slshape (Universidad Aut\'onoma de San Luis Potos\'i (UASLP))}\\
\noindent {\it  Coautor: Mar\'ia Del Carmen Rodr\'iguez Vallarte        }\\
\noindent Mostraremos las similitudes entre las \'algebras y super\'algebras de Lie, enunciaremos los resultados conocidos en cuanto a la clasificaci\'on de las \'algebras de Lie y mostraremos c\'omo a partir de esta informaci\'on se podr\'ia proceder a clasificar familias suficientemente grandes de super\'algebras de Lie.
%%%%%%%%%%%%800%%%%%%%%%%%
\subsection{\sffamily \'Algebras de Lie de Heisenberg con derivaci\'on {\footnotesize (CDV, 2Lic)}} \label{reg-800} \index{rodriguez vallarte maria del carmen@Rodr\'iguez Vallarte Mar\'ia del Carmen!\ref{reg-800}}
\noindent {\bfseries Mar\'ia del Carmen Rodr\'iguez Vallarte}, {\tt mcvallarte@gmail.com}  {\slshape ( Universidad Aut\'onoma de San Luis Potos\'i (UASLP)  Facultad de Ciencias  Departamento de Matem\'aticas )}\\
\noindent {\it  Coautor: Gil  Salgado Gonz\'alez        }\\
\noindent En esta charla trabajaremos con el \'algebra de Lie de Heisenberg $\mathfrak h$ de dimensi\'on tres, que es un \'algebra determinada por una relaci\'on de conmutaci\'on no trivial en t\'erminos de sus generadores y que adem\'as  es el \'algebra de Lie m\'as peque\~na que tiene la propiedad de ser soluble y nilpotente (es decir, ciertos productos anidados de los generadores se anulan en alg\'un momento). En analog\'{\i}a al caso semisimple, nos interesa determinar si el \'algebra soluble $\mathfrak h$ admite formas bilineales sim\'etricas o antisim\'etricas no degeneradas, que de alguna forma hagan que el corchete de Lie sea asociativo. Puesto que se verifica que esto no es posible, el siguiente paso es extender $\mathfrak h$ mediante sus derivaciones de tal manera que admita la forma bilineal en cuesti\'on. Una vez hecho esto veremos c\'omo se define la forma bilineal, determinaremos si es \'unica hasta m\'ultiplos escalares, cu\'ando dos de estas \'algebras son isomorfas y cu\'ando son isom\'etricas.  La pl\'atica ser\'a autocontenida, ejemplificaremos todos los conceptos y usando herramientas de \'algebra lineal veremos c\'omo responder las preguntas planteadas en el p\'arrafo anterior.
%%%%%%%%%%%%948%%%%%%%%%%%
\subsection{\sffamily Cohomolog\'ia de De Rham y dos aplicaciones {\footnotesize (RT, 2Lic)}} \label{reg-948} \index{mesino nunez luis alberto@Mesino N\'u\~nez Luis Alberto!\ref{reg-948}}
\noindent {\bfseries Luis Alberto Mesino N\'u\~nez}, {\tt luismenn@gmail.com}  {\slshape (Unidad Acad\'emica de Matem\'aticas de la UAgro. (U.A.M))}\\
          \noindent Construimos el producto exterior de cualquier espacio vectorial y decimos que es la derivada exterior y con ello definimos los grupos de cohomolog\'ia de De Rham para conjuntos abiertos en $\mathbb{R}^n$, para as\'i en conjunci\'on con la teor\'ia de homotop\'ia demostrar dos resultados aparentemente sin ninguna relaci\'on con la cohomolog\'ia de De Rham.
%%%%%%%%%%%%660%%%%%%%%%%%
\subsection{\sffamily El espacio de juegos como representaci\'on para el grupo sim\'etrico {\footnotesize (RT, Pos)}} \label{reg-660} \index{muniz colorado humberto alejandro@Mu\~niz Colorado Humberto Alejandro!\ref{reg-660}}
\noindent {\bfseries Humberto Alejandro Mu\~niz Colorado}, {\tt alfrednvl@gmail.com}  {\slshape (Universidad Aut\'onoma de San Luis Potos\'i (Uaslp))}\\
          \noindent En esta pl\'atica se presenta una aplicaci\'on de la teor\'ia de representaciones a soluciones en teor\'ia de juegos cooperativos. En particular, se obtiene una descomposici\'on del espacio de juegos en forma de funci\'on caracter\'istica bajo la acci\'on del grupo sim\'etrico $S_{n}$. Tambi\'en se identifican todos los subespacios irreducibles que son importantes para el estudio de soluciones lineales y sim\'etricas (i.e., aquellos que son isomorfos a los sumandos irreducibles en $ \mathbb{C}^{n}$). Por \'ultimo, se utiliza tal descomposici\'on para caracterizar todas las soluciones lineales, sim\'etricas y eficientes.
%%%%%%%%%%%%814%%%%%%%%%%%
\subsection{\sffamily Descomposiciones asociadas a sistemas de ra\'ices  en \'algebras de Lie solubles {\footnotesize (RT, 2Lic)}} \label{reg-814} \index{dorado aguilar eloy emmanuel@Dorado Aguilar Eloy Emmanuel!\ref{reg-814}}
\noindent {\bfseries Eloy Emmanuel Dorado Aguilar}, {\tt eloy10\_5@hotmail.com}  {\slshape (Universidad Aut\'onoma de San Luis Potos\'i (UASLP))}\\
          \noindent  Sea $\mathfrak{g}$ un \'algebra de Lie soluble no nilpotente con una m\'etrica invariante $ \Phi : \mathfrak{g} \times \mathfrak{g} \longrightarrow \mathbb{F}$. Por ser $ \mathfrak{g}$ soluble y tener una m\'etrica invariante, sabemos que el centro de $ \mathfrak{g}$, $ Z( \mathfrak{g} ) $, es no trivial, ?`esto nos dir\'a algo sobre la dimensi\'on de $\mathfrak{g}$?, ?`Como ser\'an los elementos de $Z(\mathfrak{g})$? Estas son algunas de las preguntas que en este trabajo de tesis se respondieron, y de esta manera podemos descomponer a $\mathfrak{g}$ y analizarla, tomando en cuenta algunos criterios, y as\'i poder aplicar los conocimientos que se tienen de las \'algebras de Lie semisimples, por ejemplo, calcular \'algebras maximales torales, espacios ra\'ices, llegando as\'i a poder dar una informaci\'on acerca de la m\'etrica $\Phi$. De manera similar se trabajo en la construcci\'on del \'algebra de Lie de Heisenberg con derivaci\'on.
%%%%%%%%%%%%815%%%%%%%%%%%
\subsection{\sffamily Descomposici\'on de \'algebras de Lie solubles que admiten m\'etricas invariantes {\footnotesize (RT, 2Lic)}} \label{reg-815} \index{martinez sigala esmeralda@Mart\'inez Sigala Esmeralda!\ref{reg-815}}
\noindent {\bfseries Esmeralda Mart\'inez Sigala}, {\tt adlae15@hotmail.com}  {\slshape (Facultad de Ciencias, Universidad Aut\'onoma de San Luis Potos\'i)}\\
          \noindent  Sea $ \mathfrak{g} $ un \'algebra de Lie. Es conocido que $\mathfrak{g}$ es semisimple s\'i, y s\'olo si la forma de Cartan-Killing $K : \mathfrak{g} \times \mathfrak{g} \longrightarrow \mathbb{F}$ es no degenerada. Ahora bien, ?`qu\'e pasa con las \'algebras de Lie solubles?. Sabemos que este criterio no nos sirve para responder esta pregunta. Tomemos $(\mathfrak{g},\Phi)$ un \'algebra de Lie cuadr\'atica soluble, no nilpotente, bajo algunos criterios que esta \'algebra nos da, podemos hacer ciertas descomposiciones, usando $\Phi$ nos dar\'a informaci\'on sobre $\mathfrak{g}$. En est\'a tesis, trabajamos en la construcci\'on del \'algebra de Lie $A_n$, la cual se obtuvo a partir de un \'algebra de Lie de dimensi\'on infinita, en este trabajo se prob\'o que $A_n$ es un \'algebra de Lie soluble, no nilpotente que admite una m\'etrica invariante $\Phi : A_n \times A_n \longrightarrow \mathbb{Z}$, y vimos hasta donde se descompone esta \'algebra con los criterios que pudimos considerar.
%%%%%%%%%%%%496%%%%%%%%%%%
\subsection{\sffamily Clases naturales y conaturales de m\'odulos {\footnotesize (RT, Pos)}} \label{reg-496} \index{garcia lopez alma violeta@Garc\'ia L\'opez Alma Violeta!\ref{reg-496}}
\noindent {\bfseries Alma Violeta Garc\'ia L\'opez}, {\tt violet1025@gmail.com}  {\slshape (Facultad de Ciencias, Universidad Nacional Aut\'onoma de M\'exico. (UNAM))}\\
          \noindent Consideraremos la clase de $R$-m\'odulos y estudiaremos subclases de \'esta dadas por ciertas propiedades de cerradura. Exploraremos las clases de m\'odulos cerradas bajo subm\'odulos, $R-her$, y principalmente las clases de m\'odulos cerradas bajo cocientes $R-quot$. Estudiaremos las ret\'iculas de  pseudocomplementos de  $R-her$ y $R-quot$ respectivamente. El estudio cl\'asico de re\'iticulas de clases de m\'odulos trata de asociar clases de m\'odulos con ciertas clases de conjuntos de ideales del anillo, para obtener entre otras consecuencias, que las clases son cardinables.  En nuestro caso, asociaremos $R-nat$ con la clase de conjuntos de ideales izquierdos que satisfacen algunas condiciones de cerradura que denotaremos como $R-Nat$. Similarmente describimos la ret\'icula cuyos elementos son filtros de ideales izquierdos. Denotamos por $R-Conat$ el esqueleto de esta ret\'icula. Obtenemos una ret\'icula cuya clase de m\'odulos asociada es una clase conatural.
%%%%%%%%%%%%1472%%%%%%%%%%%
\subsection{\sffamily Funtores y ret\'iculas de clases naturales {\footnotesize (RT, Pos)}} \label{reg-1472} \index{lopez cafaggi guillermo andres@L\'opez Cafaggi Guillermo Andr\'es!\ref{reg-1472}}
\noindent {\bfseries Guillermo Andr\'es L\'opez Cafaggi}, {\tt glopezcafaggi@gmail.com}  {\slshape (Universidad Nacional Aut\'onoma de M\'exico (UNAM))}\\
          \noindent Dentro del estudio de anillos y categor\'ias de m\'odulos se estudi\'o el concepto de clase natural; para un anillo asociativo con uno se define una clase natural como una clase de m\'odulos sobre el anillo que es cerrada bajo sumas directas, subm\'odulos y c\'apsulas inyectivas. El estudio de clases naturales y la ret\'icula de clases naturales se ha usado para definir dimensiones y descomposiciones para m\'odulos. Aqu\'i se estudia c\'omo la clase de m\'odulos de torsi\'on de goldie, que resulta una clase natural, se comporta bajo funtores definido por las ret\'iculas de clases naturales entre categor\'ias de m\'odulos y c\'omo  tambi\'en dan una descomposici\'on de la ret\'icula de clases naturales.
%%%%%%%%%%%%1455%%%%%%%%%%%
\subsection{\sffamily Localizaciones Bilaterales {\footnotesize (RT, Pos)}} \label{reg-1455} \index{medina barcenas mauricio gabriel@Medina B\'arcenas Mauricio Gabriel!\ref{reg-1455}}
\noindent {\bfseries Mauricio Gabriel Medina B\'arcenas}, {\tt mauricio\_g\_mb@yahoo.com.mx}  {\slshape (Universidad Nacional Aut\'onoma de M\'exico)}\\
          \noindent Una t\'ecnica muy usada en la teor\'ia de anillos y m\'odulos es la localizaci\'on. Esta t\'ecnica surge de querer construir a partir de un anillo $R$ otro anillo $K$ tal que $K$ tenga como subanillo a $R$ y los elementos de $R$ sean invertibles o un subconjunto de $R$ sea invertible. La respuesta a esto es el anillo de fracciones de un anillo $R$ respecto a un subconjunto multiplicativo $S\subset{R}$. El anillo de fracciones no siempre existe. \\Esta construcci\'on del anillo de fracciones y del m\'odulo de fracciones se generaliza tomando filtros de Gabriel de ideales izquierdos de un anillo $R$, los cuales sabemos est\'an en correspondencia biyectiva con las teor\'ias de torsi\'on hereditarias para $R$. Dado un filtro de Gabriel de ideales izquierdos de un anillo $R$ y un $R$-m\'odulo $M$ se construye el m\'odulo de cocientes de $M$ respecto a $\mathcal{F}$ denotado $_\mathcal{F}M$. \\El m\'odulo de cocientes respecto a un filtro de Gabriel $\mathcal{F}$ generaliza la construcci\'on del m\'odulo de fracciones. Una atenci\'on especial se le da a tomar el m\'odulo de cocientes de $_RR$ respecto a la topolog\'ia densa de $R$ y a este m\'odulo de cocientes, que resulta un anillo que tiene como subanillo a $R$, se le llama el anillo m\'aximo de cocientes de $R$. Un resultado importante respecto a este anillo es que si $R$ y $S$ son anillos Morita equivalentes entonces sus respectivos anillos m\'aximos de cocientes tambi\'en son Morita equivalentes. \\ En este trabajo se pretende dar unos resultados que generalizan el anterior, y para esto se introduce la localizaci\'on bilateral que generaliza a los m\'odulos de cocientes.
%%%%%%%%%%%%1391%%%%%%%%%%%
\subsection{\sffamily Lazos suaves, sus ecuaciones diferenciales y la geometr\'ia  correspondiente {\footnotesize (CI, Inv)}} \label{reg-1391} \index{tavdishvili larissa sbitneva@Tavdishvili Larissa Sbitneva!\ref{reg-1391}}
\noindent {\bfseries Larissa Sbitneva Tavdishvili}, {\tt larissa@uaem.mx}  {\slshape (Universidad Aut\'onoma del Estado de Morelos (UAEM) Facultad de Ciencias)}\\
          \noindent The original approach of S. Lie for Lie groups on the basis of differential equations being applied to smooth loops has permitted the development of the infinitesimal theory of smooth loops generalizing Lie groups theory ([1]) A loop with the  identity of associativity is a group. It is well known that the system of differential equations characterizing a smooth loop with the right Bol identity and the integrability conditions leads to the binary-ternary algebra as a proper infinitesimal object, which turns out to be the Bol algebra (i.e. a Lie  triple with an additional bilinear skew-symmetric operation). The corresponding geometry is related to homogeneous spaces similar to symmetric spaces and can be described in terms of the  tensors of curvature and torsi\'on ([1]). There exists the analogous consideration for Moufang loops. (1) We will consider the differential equations of smooth loops, generalizing smooth Bol loops, with the identities that are the characteristic identities for the algebraic description of some relativistic space-time models ([2]). Further  examination  of the integrability conditions for the differential equations  allows  to introduce proper infinitesimal objects for the class of loops under consideration [3]. This development leads to a Lie algebra $\frak g$ and a subalgebra $\frak h\subset \frak g$ with the decomposition$\frak g=\frak h\dot+\frak m$. The geometry of corresponding homogeneous  spaces can be described in terms of tensors of curvature and torsi\'on. There is  a relation to the notion of a left Bol loop action, since,  in the smooth case, a left Bol loop action coincides with the local triple Lie family  of Nono.\\ 1. Lev V. Sabinin: {\em Smooth Quasigroups and Loops}.   Kluwer Academic Publishers, Dordrecht, The Netherlands, 1999.\\ 2.  A. Ungar: {\em Thomas Precession:  Its Underlying Gyrogroup Axioms and Their Use in Hyperbolic Geometry and Relativistic Physics}.  Foundations of Physics. {\bf 27}, 881--951 (1997)\\ 3. L. Sbitneva: {\em $M$-loops and transsymmetric spaces}.  Aportaciones Matem\'aticas. SMM. {\bf 27}, 77-86 (2007).
%%%%%%%%%%%%870%%%%%%%%%%%
\subsection{\sffamily Grupos nilpotentes a partir de curvas anudadas {\footnotesize (CI, Pos)}} \label{reg-870} \index{mostovoy jacob@Mostovoy Jacob!\ref{reg-870}}
\noindent {\bfseries Jacob  Mostovoy}, {\tt jacob@math.cinvestav.mx}  {\slshape (Departamento de Matem\'aticas CINVESTAV)}\\
          \noindent El tema de esta charla es un tipo de grupos nilpotentes que se construyen como clases de equivalencia de ciertas curvas anudadas (conocidas en ingl\'es como ``string links''). Hablar\'e del contexto general donde aparecen estos grupos y de los problemas que siguen abiertos en el campo.
%%%%%%%%%%%%419%%%%%%%%%%% Cancelada
%\subsection{\sffamily Grupos casi c\'iclicos {\footnotesize (RT, 2Lic)}} \label{reg-419} \index{rodriguez olivarez miriam@Rodr\'iguez Olivarez Miriam!\ref{reg-419}}
%\noindent {\bfseries Miriam  Rodr\'iguez Olivarez}, {\tt magic\_mro@hotmail.com}  {\slshape (Universidad Veracruzana (UV))}\\
 %         \noindent Los grupos casi c\'iclicos por su definici\'on los  podemos considerar  como una generalizaci\'on de los grupos c\'iclicos, y en la pl\'atica que se dar\'a  veremos algunas propiedades que llevan los grupos casi c\'iclicos a ser  c\'iclicos, adem\'as de algunos ejemplos y teoremas que caracterizan a los  grupos casi c\'iclicos.
%%%%%%%%%%%%1161%%%%%%%%%%%
\subsection{\sffamily N\'umeros de Betti de ideales monomiales {\footnotesize (CPI, Pos)}} \label{reg-1161} \index{martinez-bernal jose@Mart\'inez-Bernal Jos\'e!\ref{reg-1161}}
\noindent {\bfseries Jos\'e  Mart\'inez-Bernal}, {\tt jmb@math.cinvestav.mx}  {\slshape (Centro de Investigaci\'on y de Estudios Avanzados (CINVESTAV))}\\
          \noindent Se presentan problemas relacionados con n\'umeros de Betti de ideales monomiales en un anillo de polinomios, as\'i como algunas de sus conexiones con estad\'istica algebraica.
%%%%%%%%%%%%1704%%%%%%%%%%%
\subsection{\sffamily \'Algebra lineal computacional sobre campos finitos {\footnotesize (CDV, 2Lic)}} \label{reg-1704} \index{lopez bautista pedro ricardo@L�pez Bautista Pedro Ricardo!\ref{reg-1704}}
\noindent {\bfseries Pedro Ricardo L\'opez Bautista}, {\tt rlopez@correo.azc.uam.mx}  {\slshape (Universidad Aut\'onoma Metropolitana-Azcapotzalco (UAM) Departamento de Ciencias B\'asicas)}\\
\noindent {\it  Coautores: Georgina  Pulido Rodr\'iguez, Galois  Rodr\'iguez \'Alvarez      }\\
\noindent En esta pl\'atica usaremos un enfoque computacional para ilustrar propiedades y problemas en \'Algebra lineal. Utilizaremos algunos CAS y librer\'{\i}as como Octave, Magma, Kash/Kant, Sage, Pari, vxMaxima, Mathematica, Geogebra, GAP, GMP, LIP, NTL.  LiDia mostrando car\'acter\'{\i}sticas fundamentales de cada uno de los CAS mencionados y ventajas de unos sobre otros. Usando estos CAS, ejemplificamos con algoritmos y pseudoc\'odigos conceptos y problemas en \'Algebra, matrices densas y matrices sparse sobre los enteros y campos finitos y damos soluci\'on a algunos problemas de \'algebra lineal sobre campos finitos.
%%%%%%%%%%%%843%%%%%%%%%%%
\subsection{\sffamily Sobre una identidad determinantal universal {\footnotesize (CI, 2Lic)}} \label{reg-843} \index{weingart gregor@Weingart Gregor!\ref{reg-843}}
\noindent {\bfseries Gregor  Weingart}, {\tt gw@matcuer.unam.mx}  {\slshape (Instituto de Matem\'aticas (Cuernavaca), Universidad Nacional Aut\'onoma de M\'exico (UNAM))}\\
          \noindent La teor\'ia de las representaciones de los grupos finitos, en particular de los grupos sim\'etricos, es uno de los grandes logros de las matem\'aticas de finales del siglo XIX. En un sentido la teor\'ia dual a la teor\'ia de las representaciones irreducibles de los grupos sim\'etricos es la teor\'ia de los llamados funtores de Schur, que aparecen en varias \'areas de las matem\'aticas. En mi pl\'atica quiero presentar una construcci\'on de las representaciones irreducibles de los grupos sim\'etricos mediante una identidad determinantal ``universal'', motivada por una construcci\'on de los funtores de Schur. Casos especiales de esta identidad determinantal universal son la f\'ormula de caracteres de Fr\"obenius y la f\'ormula de caracteres de Weyl por los grupos de Lie cl\'asicos.
%%%%%%%%%%%%443%%%%%%%%%%%
\subsection{\sffamily Persistencia de Ideales de Gr\'aficas {\footnotesize (CI, 2Lic)}} \label{reg-443} \index{toledo jonathan@Toledo Jonathan!\ref{reg-443}}
\noindent {\bfseries Jonathan  Toledo}, {\tt jtt@math.cinvestav.mx}  {\slshape (Departamento de Matem\'aticas del Centro de Investigaci\'on y Estudios Avanzados (CINVESTAV-IPN))}\\
          \noindent Decimos que un ideal $I$ de un anillo $A$ cumple la propiedad de persistencia si la colecci\'on de primos asociados de las potencias de dicho ideal forma una cadena ascendente, esto es $\texttt{\rm Ass}(I^k) \subseteq \texttt{\rm Ass}(I^{k+1}) $ para cada $k$, donde $\texttt{\rm Ass}(I^k)=\{P \in \texttt{\rm Spec}(A) \mid \exists \ a \in A \ \texttt{\rm tal que}\  P=(I^k \colon a) \}$. Se conoce varias clases de ideales que cumplen dicha propiedad, como los ideales normales, clase que incluye a los ideales de gr\'aficas que cumplen la condici\'on ciclo impar, recientemente esto se generaliz\'o a cualquier gr\'afica, por lo que nosotros nos dedicamos a averiguar si tambi\'en ocurre esto para gr\'afica con loops y gr\'aficas pesadas, se encontr\'o que para gr\'aficas con loops se cumple la propiedad al igual que para varios casos de gr\'aficas pesadas.
%%%%%%%%%%%%1813%%%%%%%%%%%
\subsection{\sffamily \'Ultima generalizaci\'on del c\'odigo Reed-Muller af\'in generalizado {\footnotesize (CI, Pos)}} \label{reg-1813} \index{sanchez antonio jesus@S\'anchez Antonio Jes\'us!\ref{reg-1813}}
\noindent {\bfseries Antonio Jes\'us S\'anchez}, {\tt fosi\_ipn@msn.com}  {\slshape (Instituto Polit\'ecnico Nacional (IPN))}\\
          \noindent Dentro de la  teor\'ia de c\'odigos existen distintos problemas a atacar para la localizaci\'on de los par\'ametros fundamentales. Recientemente han surgido distintos tipos de enfoques que garantizan una nueva generalizaci\'on de c\'odigos conocidos. En este trabajo se presenta el ejemplo del c\'odigo Reed-Muller af\'in que fue resulto en la d\'ecada de los sesentas y ahora se retoma para la  comprensi\'on de nuevos c\'odigos.
