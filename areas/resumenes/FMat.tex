\section{F\'isica Matem\'atica y Geometr\'ia Diferencial}

%%%%%%%%%%%%1514%%%%%%%%%%%
\subsection{\sffamily Modelado matem\'atico de sistemas biol\'ogicos complejos {\footnotesize (CPI, 2Lic)}} \label{reg-1514} \index{Hernandez Lemus Enrique@Hern\'andez Lemus Enrique!\ref{reg-1514}}
\noindent {\bfseries Enrique  Hern\'andez Lemus}, {\tt ehernandez@inmegen.gob.mx}  {\slshape (Instituto Nacional de Medicina Gen\'omica (INMEGEN) Departamento de Gen\'omica Computacional)}\\
          \noindent La investigaci\'on contempor\'anea en sistemas biol\'ogicos se est\'a tornando m\'as y m\'as proclive al modelado matem\'atico y probabil\'istico. Con el advenimiento de nuevas tecnolog\'ias que generan datos masivos -en gen\'omica, prote\'omica y neurociencias, entre otras- aunado al enfoque integrador conocido como biolog\'ia de sistemas; han surgido una serie de problemas que involucran el empleo y la generaci\'on de nuevos enfoques de an\'alisis matem\'atico que tradicionalmente se han asociado con la f\'isica y en particular con la f\'isica matem\'atica: los sistemas din\'amicos lineales y no-lineales, la teor\'ia erg\'odica, la teor\'ia de grafos, los enfoques bayesiano y de m\'axima entrop\'ia para estudiar distribuciones de probabilidad, muchas veces en soportes no-markovianos, etc. En esta pl\'atica comentaremos como es posible emplear algunas de estas t\'ecnicas matem\'aticas -e incluso como, en muchos casos, ha sido necesario desarrollar nueva matem\'atica- en el an\'alisis de sistemas biol\'ogicos, particularmente en aquellos de inter\'es biom\'edico.
%%%%%%%%%%%%984%%%%%%%%%%%
\subsection{\sffamily C\'alculo y aplicaci\'on de la discretizaci\'on del Operador Laplace-Beltrami en la automatizaci\'on de An\'alisis de Im\'agenes {\footnotesize (CDV, 2Lic)}} \label{reg-984} \index{Martinez Vega Rafael@Mart\'inez Vega Rafael!\ref{reg-984}}
\noindent {\bfseries Rafael  Mart\'inez Vega}, {\tt rafael.martinez@uacm.edu.mx}  {\slshape (Academia de Matem\'aticas, Universidad Aut\'onoma de la Ciudad de M\'exico (UACM)  Department of Mathematics Florida State University (FSU))}\\
          \noindent El an\'alisis de im\'agenes m\'edicas en a\~nos recientes ha requerido la aplicaci\'on de conceptos de geometr\'ia diferencial para lograr automatizar la detecci\'on de diferencias entre superficies de inter\'es. El espectro de los valores propios del Operador Laplace-Beltrami, utilizado para resolver la ecuaci\'on de calor- se han vuelto de gran utilidad en la detecci\'on de detalles finos de las superficies de inter\'es.  De esta forma se pretende automatizar el registro (comparaci\'on) de diversas superficies, que representan \'organos de inter\'es por ejemplo. En esta pl\'atica se discutir\'a c\'omo es que se calcula el Kernel de Calor para hallar los valores propios del Operador Laplace-Beltrami sobre representaciones discretas de superficies. En particular tratar\'a de las t\'ecnicas descritas en el art\'iculo ``Discrete Heat Kernel determines Discrete Riemannian Metric'' de Zeng et al. 2012, y tambi\'en se discutir\'a brevemente sus aplicaciones en an\'alisis de im\'agenes m\'edicas.
%%%%%%%%%%%%1803%%%%%%%%%%%
\subsection{\sffamily Estudio comparativo de la microhidrataci\'on de las bases de los \'acidos nucleicos, usando m\'etodos de mec\'anica molecular y mec\'anica cu\'antica {\footnotesize (RI, Pos)}} \label{reg-1803} \index{Lino Perez Job Israel@Lino P\'erez Job Israel!\ref{reg-1803}}
\noindent {\bfseries Job Israel Lino P\'erez}, {\tt jlino\_x@yahoo.com.mx}  {\slshape (Benem\'erita Universidad Aut\'onoma de Puebla (BUAP))}\\
          \noindent El ADN es una mol\'ecula esencial para el funcionamiento biol\'ogico de un organismo vivo, en la cual la hidrataci\'on es primordial para la estructura y funcionabilidad de la doble h\'elice. La microhidrataci\'on de las bases individuales de los \'acidos nucleicos y sus derivados metilados, se analizan utilizando m\'etodos de Mec\'anica Molecular (MM) con los campos de fuerzas de Poltev-Malenkov, AMBER y Joergensen, y con Mec\'anica Cu\'antica (MC) con c\'alculos ab initio MP2/6-31G(d,p). La validaci\'on de los resultados num\'ericos se hace comparando con los datos experimentales de la entalpia de la microhidrataci\'on de las bases, obtenidos a partir de espectroscopia de masas a bajas temperaturas. Cada m\'inimo local de una mol\'ecula de agua con las bases de los AN obtenido con MM tiene su correspondencia con MC,  se observ\'o  en general una concordancia cualitativa en la geometr\'ia de los m\'inimos locales, con los potenciales de MM se ven ligeramente m\'as favorecidas las estructuras de tipo coplanar, sus valores de energ\'ia en valor absoluto sobrevaloran los de MC. Para Adenina y Timina lo valores en los m\'inimos locales est\'an m\'as cercanos con el potencial PM (0.72 kcal/mol) que AMBER (1.86 kcal/mol). Los cambios energ\'etico m\'as marcados respecto a MC son para Guanina y Citosina, principalmente en m\'inimos donde el agua forma 2 enlaces-H con dos centros prot\'on-aceptor de la base (4.26 y 3.52 kcal/mol respectivamente para PM) este es el m\'inimo mas profundo en MM. En cambio de los c\'alculos de MC se ve que el m\'inimo global es cuando la mol\'ecula de agua forma 2 enlaces-H con  un centro donador y uno aceptor. Los c\'alculos con las bases trimetiladas con una mol\'ecula de agua corroboran este hecho. Estos datos al igual que los perfiles de energ\'ia obtenidos para los monohidratos cuando se fijan algunos par\'ametros en la mol\'ecula de agua contribuir\'an al mejoramiento y ajuste del campo de fuerzas de mec\'anica molecular.
%%%%%%%%%%%%1530%%%%%%%%%%%
\subsection{\sffamily Invariante Modular del Toro Cu\'antico {\footnotesize (CI, Inv)}} \label{reg-1530} \index{Gendron Timothy@Gendron Timothy!\ref{reg-1530}}
\noindent {\bfseries Timothy  Gendron}, {\tt tim@matcuer.unam.mx}  {\slshape (Instituto de Matem\'aticas Unidad Cuernavaca UNAM)}\\
          \noindent En esta charla definiremos un invariante modular universal que toma valores en una $\mathbb{C}$-\'{a}lgebrade transversales de un solenoide no-est\'andar.  Invariantes modulares cl\'{a}sicos y cu\'{a}nticos  $\diamond \hat{\jmath}^{\rm cl}$ y ${}^{\diamond} \hat{\jmath}^{\rm qt}$ son definidos como restricciones subtransversales: el primero una funci\'{o}n de la curva modular y el segundo una funci\'{o}n de su haz tangente unitario.  Para $\mu\in\mathbb{H}$, ${}^{\diamond}\hat{\jmath}^{\rm cl}(\mu)$ es asint\'{o}tica al invariante modular usual $j(\mu )$; para $\theta\in\mathbb{R}\cup\{\infty\}\approx$ el espacio tangente unitario de $i$, ${}^{\diamond}\hat{\jmath}^{\rm qt}(i,\theta )$ es asint\'{o}tico a un limite $j^{\rm qt}(\theta )$de expresiones est\'andares definido usando la funci\'{o}n ``distancia al entero mas cercano''. En el caso de $\theta=\varphi$ = la raz\'on \'{a}urea, se demuestra que $j^{\rm qt}(\varphi )\approx 9538.249655644 $ al usar una f\'{o}rmula expl\'{\i}cita para $j^{\rm qt}(\varphi )$ que involucra generalizaciones con pesos de las funciones de Rogers-Ramanujan.
%%%%%%%%%%%%729%%%%%%%%%%%
\subsection{\sffamily Teor\'ia de dispersi\'on cu\'antica  para potenciales sencillos {\footnotesize (CPI, 2Lic)}} \label{reg-729} \index{Cruz Sampedro Jaime@Cruz Sampedro Jaime!\ref{reg-729}}
\noindent {\bfseries Jaime  Cruz Sampedro}, {\tt cruzsampedro@gmail.com}  {\slshape (\'Area de An\'alisis Matem\'atico de la Universidad Aut\'onoma Metropolita\-na-Azcapotzalco (UAM-A))}\\
          \noindent El  fen\'omeno de  dispersi\'on se manifiesta en situaciones muy diversas. Con frecuencia, la manera m\'as efectiva  de estudiar la naturaleza microsc\'opica de objetos peque\~nos o inaccesibles (o de establecer su estructura y posici\'on) es mediante  la dispersi\'on de ondas o part\'iculas. Un experimento t\'ipico de dispersi\'on es el Experimento de Rutherford (1871-1937),   con el que se  sustent\'o el modelo planetario del n\'ucleo at\'omico. En la primera parte de esta charla describimos de manera  general e intuitiva el fen\'omeno de dispersi\'on. Luego   planteamos f\'isica y matem\'aticamente  el problema de dispersi\'on de la mec\'anica cu\'antica.  Finalmente describimos  resultados cl\'asicos y algunas contribuciones del autor en este campo para potenciales sencillos. La primera parte de esta charla es accesible para todo el mundo. Para la segunda es deseable cierta familiaridad con  los conceptos b\'asicos del an\'alisis matem\'atico, un poco de an\'alisis funcional y algo de ecuaciones diferenciales.
%%%%%%%%%%%%1154%%%%%%%%%%%
\subsection{\sffamily Soluci\'on de un Problema de \'Optica Cu\'antica Usando Teor\'ia  de Semigrupos {\footnotesize (CI, Pos)}} \label{reg-1154} \index{Gonzalez Gaxiola Oswaldo@Gonz\'alez Gaxiola Oswaldo!\ref{reg-1154}}
\noindent {\bfseries Oswaldo  Gonz\'alez Gaxiola}, {\tt ogonzalez@correo.cua.uam.mx}  {\slshape (Depto. de Matem\'aticas Aplicadas y Sistemas, UAM-Cuajimalpa (UAM-C))}\\
          \noindent Vamos a considerar el Hamiltoniano que resulta de la interacci\'on de un oscilador arm\'onico cu\'antico con un rayo de luz no-cl\'asica (L\'aser) como un operador actuando sobre un cierto espacio de Hilbert; adem\'as consideraremos que dicho sistema evoluciona en presencia de una fuerza externa y haciendo uso de la teor\'ia desemigrupos de operadores; estableceremos la soluci\'on (d\'ebil) del problema.
%%%%%%%%%%%%1389%%%%%%%%%%%
\subsection{\sffamily Geometr\'ia de superficies: aplicaciones {\footnotesize (CI, Pos)}} \label{reg-1389} \index{Santiago Jose Antonio@Santiago Jos\'e Antonio!\ref{reg-1389}}
\noindent {\bfseries Jos\'e Antonio Santiago}, {\tt jasantiagog@gmail.com}  {\slshape (Departamento de Matem\'aticas Aplicadas. Universidad Aut\'onoma Metropolitana Cuajimalpa (UAM-C))}\\
          \noindent Daremos un breve resumen de la teor\'ia geom\'etrica de superficies, en t\'erminos de la primera y la segunda forma fundamentales. Motivaremos brevemente el funcional de doblamiento, proporcional al cuadrado de la curvatura media de la superficie. Encontraremos las ecuaciones que optimizan esta energ\'ia para una membrana en un medio con viscosidad y finalmente abordaremos el problema de la estabilidad alrededor de estas soluciones. El an\'alogo para curvas con rigidez ser\'a mencionado brevemente.
%%%%%%%%%%%%1319%%%%%%%%%%%
\subsection{\sffamily Materiales con memoria de forma - Transiciones de fase coherentes {\footnotesize (RT, 2Lic)}} \label{reg-1319} \index{Caballero Altamirano Arturo@Caballero Altamirano Arturo!\ref{reg-1319}}
\noindent {\bfseries Arturo  Caballero Altamirano}, {\tt caballero000@gmail.com}  {\slshape (Centro de Investigaciones en Matem\'aticas (CIMAT))}\\
          \noindent Muchos sistemas f\'isicos pueden ser modelados por sistemas variacionales no convexos regularizados por t\'erminos de alto orden. Ejemplos pueden encontrarse en transformaciones de fase martens\'iticas, micromagnetismo, entre otros. Gran parte estos estudios han sido motivados por el efecto de memoria de forma que se presenta en aleaciones de metales como el Nitinol. Durante la charla abordaremos un modelo variacional para este tipo de materiales y las microestructuras que se observan.
%%%%%%%%%%%%486%%%%%%%%%%%
\subsection{\sffamily Modelado del Abordaje de Aviones V\'ia Geometr\'ia del Espacio-Tiempo {\footnotesize (RT, 2Lic)}} \label{reg-486} \index{Rios Hernandez Ana Sofia@R\'ios Hern\'andez Ana Sof\'ia!\ref{reg-486}}
\noindent {\bfseries Ana Sof\'ia R\'ios Hern\'andez}, {\tt asofia.rios@gmail.com}  {\slshape (Universidad Veracruzana (UV), Facultad de Matem\'aticas)}\\
\noindent {\it  Coautores: Didier Ad\'an Sol\'is Gamboa, Francisco Gabriel Hern\'andez Zamora      }\\
\noindent El proceso de abordaje de un avi\'on es llevado a cabo todos d\'ias por millones de pasajeros alrededor del mundo. Las aerol\'ineas tienen que recurrir a diversas estrategias de abordaje con la esperanza de disminuir el tiempo que les toma a los pasajeros ingresar al avi\'on y sentarse,  despejando as\'i la sala de \'ultima espera. La estrategia m\'as popular en la actualidad es la implementada por los anuncios de la forma ``Pasajeros de la fila 30 y m\'as son bienvenidos a abordar el avi\'on''. Sin embargo, no existen estudios contundentes que garanticen que esta estrategia reduce significativamente el tiempo de abordaje. El prop\'osito de esta charla es describir el modelado del proceso de abordaje usando herramientas geom\'etricas, espec\'ificamente la geometr\'ia de Lorentz, tambi\'en conocida como geometr\'ia del espacio-tiempo. Como resultado de este an\'alisis se encontr\'o el tiempo esperado para  culminar el abordaje de un avi\'on Boeing 737-300 en ausencia de una pol\'itica de abordaje pre-establecida (es decir, abordaje aleatorio). Cabe decir que este valor es de suma importancia, ya que brinda una primera pauta para establecer el \'exito de una pol\'itica de abordaje dada.
%%%%%%%%%%%%1366%%%%%%%%%%%
\subsection{\sffamily Entrop\'ia a lo largo del flujo de Yamabe {\footnotesize (CI, Inv)}} \label{reg-1366} \index{Suarez Serrato Pablo@Su\'arez Serrato Pablo!\ref{reg-1366}}
\noindent {\bfseries Pablo  Su\'arez Serrato}, {\tt ps358@matem.unam.mx}  {\slshape (IMATE DF)}\\
          \noindent Explicaremos como cambian la entrop\'ia volum\'etrica y topol\'ogica a lo largo del un flujo de Yamabe normalizado respecto a curvatura. Empezando desde una m\'etrica de curvatura escalar negativa y acotada, las entrop\'ias est\'an acotadas por los valores de las entrop\'ias de la m\'etrica de curvatura escalar constante -1. Estos resultados son v\'alidos para variedades lisas compactas y tambi\'en para algunas variedades no-compactas de volumen infinito, conocidas como variedades convexas cocompactas (de curvatura negativa variable).  Estos trabajos son parte de una colaboraci\'on con el Dr. Samuel Tapie de la Universidad de Nantes, Francia.
%%%%%%%%%%%%773%%%%%%%%%%%
\subsection{\sffamily Grupos cu\'anticos y geometr\'ia no-conmutativa {\footnotesize (CPI, 2Lic)}} \label{reg-773} \index{Wagner Elmar@Wagner Elmar!\ref{reg-773}}
\noindent {\bfseries Elmar  Wagner}, {\tt elmar@ifm.umich.mx}  {\slshape ( Instituto de F\'isica y Matem\'aticas (IFM),  Universidad Michoacana de San Nicol\'as de Hidalgo (UMSNH) )}\\
          \noindent Los grupos cu\'anticos y la geometr\'ia no-conmutativa de Connes representan un programa de reformulaci\'on de la teor\'ia de grupos y \'algebras de Lie, y la topolog\'ia y geometr\'ia diferencial, respectivamente, con m\'etodos algebro-geom\'etricos. El \'exito de esas teor\'ias resulta de sus amplios relaciones y aplicaciones en diferentes \'areas de matem\'aticas y de sus interfaces con la f\'isica te\'orica, tales como la estructura del espacio-tiempo en distancias extremadamente peque\~nas,  el modelo est\'andar de f\'isica de part\'iculas, el efecto Hall cu\'antico, la teor\'ia de cuerdas, los modelos integrables cu\'anticos y las teor\'ias de campos conformes. El objetivo de la pl\'atica es de explicar, mediante un simple ejemplo, los primeros pasos para construir una teor\'ia de gauge sobre espacios cu\'anticos.
%%%%%%%%%%%%1566%%%%%%%%%%%
\subsection{\sffamily Teoremas de separaci\'on en geometr\'ia lorentziana {\footnotesize (CPI, Pos)}} \label{reg-1566} \index{Solis Gamboa Didier Adan@Sol\'is Gamboa Didier Ad\'an!\ref{reg-1566}}
\noindent {\bfseries Didier Ad\'an Sol\'is Gamboa}, {\tt didier.solis@uady.mx}  {\slshape (Facultad de Matem\'aticas. Universidad Aut\'onoma de Yucat\'an (UADY))}\\
          \noindent Los teoremas de separaci\'on surgen en el contexto de la geometr\'ia riemanniana, siendo el Teorema de Cheeger-Gromoll el m\'as famoso de ellos. En t\'erminos muy generales, un teorema de separaci\'on establece que bajo ciertas condiciones de curvatura, geod\'esicas que relizan distancia entre cualesquiera dos de sus puntos s\'olo pueden existir cuando la variedad en cuesti\'on es un producto.  En esta charla describiremos los principales teoremas de separaci\'on que existen en geometr\'ia lorentziana y algunas aplicaciones recientes de los mismos.
%%%%%%%%%%%%1443%%%%%%%%%%%
\subsection{\sffamily Geometr\'\i a de la Informaci\'on, Variedades Gamma {\footnotesize (CDV, 2Lic)}} \label{reg-1443} \index{Del Riego Senior Lilia Maria@Del Riego Senior Lilia Mar\'ia!\ref{reg-1443}}
\noindent {\bfseries Lilia Mar\'ia Del Riego Senior}, {\tt lilia@fc.uaslp.mx}  {\slshape (Departamento de Matem\'aticas Facultad de Ciencias Universidad Aut\'onoma de San Luis Potos\'i (UASLP))}\\
          \noindent La geometr\'ia de la informaci\'on es una nueva rama de las matem\'aticas que aplica t\'ecnicas de la geometr\'\i a diferencial al campo de la teor\'\i a de la probabilidad. Las distribuciones de la probabilidad de un modelo estad\'\i stico son los puntos de una variedad Riemanniana, La m\'etrica utilizada se llama de Fisher. En esta pl\'atica se introducir\'an conceptos geom\'etricos y topol\'ogicos que surgen en las variedades asociadas a las distribuciones probabil\'\i sticas Gamma.
%%%%%%%%%%%%1722%%%%%%%%%%%
\subsection{\sffamily Estructuras Riemannianas en la Termodin\'amica {\footnotesize (RT, Inv)}} \label{reg-1722} \index{Garcia Ariza Miguel Angel@Garc\'ia Ariza Miguel \'Angel!\ref{reg-1722}}
\noindent {\bfseries Miguel \'Angel  Garc\'ia Ariza}, {\tt magarciaariza@gmail.com}  {\slshape (Benem\'erita Universidad Aut\'onoma de Puebla (BUAP))}\\
\noindent {\it  Coautor: Mar\'ia del Roc\'io  Mac\'ias Prado        }\\
\noindent En esta pl\'atica se presenta el uso de la geometr\'ia diferencial en la termodin\'amica cl\'asica. Se comienza con una construcci\'on del Espacio Fase Termodin\'amico similar a la de Carath\'eodory y se introducen estructuras riemannianas en \'este. Se mostrar\'an algunas de las utilidades de este formalismo, as\'i como las interrogantes que a\'un presenta.
%%%%%%%%%%%%308%%%%%%%%%%%
\subsection{\sffamily Transformada de Penrose sobre D-M\'odulos, Espacios Moduli y Teor\'ia de Campo {\footnotesize (RI, Inv)}} \label{reg-308} \index{Bulnes Aguirre Francisco@Bulnes Aguirre Francisco!\ref{reg-308}}
\noindent {\bfseries Francisco  Bulnes Aguirre}, {\tt francisco.bulnes@tesch.edu.mx}  {\slshape (Departamento de Investigaci\'on en Matem\'aticas e Ingenier\'ia, Tecnol\'ogico de Estudios Superiores Chalco (DIMI-TESCHA))}\\
          \noindent  Consideramos una generalizaci\'on de la transformada de Radon-Schmid sobre D-m\'odulos coherentes de gavillas de haces holomorfos complejos dentro de un espacio moduli con el prop\'osito de establecer las equivalencias entre objetos geom\'etricos (los haces vectoriales) y los objetos algebraicos que utilizamos, los D-m\'odulos coherentes, \'estos \'ultimos con el objetivo de obtener clases conformes de conexiones de los h\'aces holomorfos complejos. La clase de estas equivalencias conforman un espacio moduli sobre gavillas coherentes que definen soluciones en teor\'ia de campo. Tambi\'en por este camino, y usando una generalizaci\'on de la transformada de Penrose en el contexto de los D-m\'odulos coherentes encontramos clases conformes del espacio-tiempo que incluyen la geometr\'ia de cuerdas heter\'oticas y branas.
%%%%%%%%%%%%403%%%%%%%%%%%
\subsection{\sffamily Teor\'ia Cu\'antica de Campos en Variedades Lorenzianas {\footnotesize (RT, Pos)}} \label{reg-403} \index{Garcia Lara Rene Israel@Garc\'ia Lara Ren\'e Israel!\ref{reg-403}}
\noindent {\bfseries Ren\'e Israel Garc\'ia Lara}, {\tt rene.garcia@uady.mx}  {\slshape (Universidad Aut\'onoma de Yucat\'an)}\\
          \noindent Este trabajo trata acerca de la ecuaci\'on de onda en un espacio curvado por el efecto de la gravedad. Si el universo es modelado como un espacio globalmente hiperb\'olico, es posible demostrar que la ecuaci\'on de onda posee soluciones globales, y que el problema de Cauchy asociado esta bien planteado. Pedir al universo que sea globalmente hiperb\'olico tiene ciertas consecuencias, como la existencia de un campo vectorial de tipo temporal futuro, que muchas veces se interpreta como la direcci\'on del tiempo, o la entrop\'ia. Es en estos espacios, que la existencia global de una soluci\'on de la ecuaci\'on de onda permite aplicar un m\'etodo para cuantizar el operador de onda, que se llama cubanizaci\'on algebraica, y que resulta ser un funtor de la categor\'ia de los espacios globalmente hiperb\'olicos a la categor\'ia de las algebras C*. En este reporte de tesis se presentan los aspectos matem\'aticos relacionados con la ecuaci\'on de onda en espacios globalmente hiperb\'olicos.
%%%%%%%%%%%%379%%%%%%%%%%%
\subsection{\sffamily Breve introducci\'on al \'algebra geom\'etrica {\footnotesize (CDV, 1Lic)}} \label{reg-379} \index{Herrera Guzman Rafael@Herrera Guzm\'an Rafael!\ref{reg-379}}
\noindent {\bfseries Rafael  Herrera Guzm\'an}, {\tt rherrera.cimat@gmail.com}  {\slshape (Centro de Investigaci\'on en Matem\'aticas, A. C.)}\\
          \noindent En esta pl\'atica introducir\'e los conceptos b\'asicos del \'algebra geom\'etrica como herramienta para estudiar situaciones geom\'etricas en 2D y 3D. Cabe se\~nalar que el \'algebra geom\'etrica NO es geometr\'ia algebraica. El t\'ermino \'algebra geom\'etrica es usado en f\'isica y ciencias de la computaci\'on, mientras que en matem\'aticas le llamamos la representaci\'on del \'algebra de Clifford en s\'i misma. El \'algebra geom\'etrica tiene aplicaciones en ciencias computacionales, f\'isica y matem\'aticas, algunas de las cuales ser\'an mencionadas brevemente.
%%%%%%%%%%%%840%%%%%%%%%%%
\subsection{\sffamily Estructuras geom\'etricas en supervariedades {\footnotesize (RI, 2Lic)}} \label{reg-840} \index{weingart gregor@Weingart Gregor!\ref{reg-840}}
\noindent {\bfseries Gregor  Weingart}, {\tt gw@matcuer.unam.mx}  {\slshape (Instituto de Matem\'aticas (Cuernavaca), Universidad Nacional Aut\'onoma de M\'exico (UNAM))}\\
          \noindent En geometr\'ia diferencial cl\'asica una estructura geom\'etrica se puede definir como una secci\'on de un haz natural sobre una variedad suave. La naturaleza del haz es muy importante en esta definici\'on, por que este concepto asocia una ecuaci\'on de Killing a la estructura geom\'etrica estudiada. En generalizaci\'on del tensor de curvatura en geometr\'ia diferencial Riemanniana la ecuaci\'on de Killing describe los invariantes locales de la estructura geom\'etrica v\'ia su cohomolog\'ia de Spencer. En mi pl\'atica quiero dar unos ejemplos cl\'asicos sobre la relaci\'on entre la ecuaci\'on de Killing, su cohomolog\'ia de Spencer y invariantes locales, en particular quiero tratar los variedades simpl\'ecticas y cas\'i complejas, y quiero dar un bosquejo de las investigaciones de mi estudiante \'Oscar Francisco Guajardo Garza y m\'i para establecer la misma relaci\'on para las estructuras geom\'etricas en supervariedades.
%%%%%%%%%%%%1057%%%%%%%%%%%
\subsection{\sffamily Cuantizaci\'on por Deformaci\'on en F\'isica y Matem\'aticas {\footnotesize (CPI, 2Lic)}} \label{reg-1057} \index{Garcia Compean Hector Hugo@Garc\'ia Compean H\'ector Hugo!\ref{reg-1057}}
\noindent {\bfseries H\'ector Hugo Garc\'ia Compean}, {\tt compean@fis.cinvestav.mx}  {\slshape (Departamento de F\'isica, Centro de Investigaci\'on y de Estudios Avanzados (CINVESTAV))}\\
          \noindent En esta platica, daremos una visi\'on panor\'amica del proceso de cuantizaci\'on utilizando la deformaci\'on asociativa y no-conmutativa de \'algebras de funciones. Enfatizaremos los principales resultados en la F\'isica y en las Matem\'aticas.


