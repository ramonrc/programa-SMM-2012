\section{Docencia}

%%%%%%%%%%%%1753%%%%%%%%%%%
\subsection{\sffamily Desarrolla competencias matemáticas jugando con material manipulable {\footnotesize (, )}} \label{reg-1753} \index{Gómez Avilés Patricia}
\noindent {\bfseries Patricia  Gómez Avilés}, {\tt patygomeza@hotmail.com}  {\slshape (SECRETARÍA DE EDUCACIÓN PÚBLICA HIDALGO. DIRECCIÓN DE INVESTIGACIÓN EDUCATIVA.)}\\
\noindent {\it  Coautor: GERSÓN  HERNÁNDEZ MARTÍNEZ        }\\
\noindent El taller “Desarrolla Competencias Matemáticas Jugando con Material Manipulable” tiene el propósito de propiciar en los participantes la resolución de problema con temas relacionados a los ejes temáticos Sentido Numérico y Pensamiento Algebraico; asimismo el de Forma, Espacio y Medida, al generar en ellos la necesidad de utilizar materiales manipulable y juegos didácticos que propone la propuesta de “Ludoteca Interactiva de Matemáticas” al realizar actividades lúdicas, que contribuyan en el desarrollo de competencias matemáticas (Resolución de Problemas de Manera Autónoma; Comunicar Información Matemática; Validar Procedimientos y Resultados; Manejar Técnicas Eficientemente) con la finalidad de que se apropien de estrategias que puedan aplicar en sus aulas para el mejoramiento del proceso enseñanza-aprendizaje.
%%%%%%%%%%%%1105%%%%%%%%%%%
\subsection{\sffamily Técnicas para el aprendizaje significativo de las matemáticas en el aula {\footnotesize (, )}} \label{reg-1105} \index{Esquivel Navarrete Anel}
\noindent {\bfseries Anel  Esquivel Navarrete}, {\tt aesquiveln@uaemex.mx}  {\slshape (Facultad de Ciencias, Universidad Autónoma del Estado de México (UAEM))}\\
\noindent {\it  Coautores: ANA CECILIA SIERRA CUEVAS, JUANA IMELDA VILLARREAL VALDÉS      }\\
\noindent Este taller presentaremos una serie de trucos mágicos con un fundamento matemático detrás. Provocaremos que el participante razone en el por qué siempre resulta el truco que proponemos e innove otras formas de presentar esos mismos trucos. Por otro lado,  se abordarán algunos sistemas antiguos de numeración, concretamente trabajaremos con el sistema de numeración maya y con el ábaco de Napier; con ellos mostraremos de manera sencilla y fascinante algunos procedimientos para realizar las operaciones matemáticas fundamentales: suma, resta y multiplicación, sin necesidad de memorizar tablas de ningún tipo sino con la ayuda de material palpable como tableros, caracoles, botones, fichas de juego, etc. Convirtiéndose todas estas técnicas en poderosas herramientas intuitivas, dinámicas y lúdicas de la enseñanza de las matemáticas. Así mismo, advertimos que estas actividades se puede utilizar en los diferentes momentos de las clases en el aula: al inicio para atraer la atención de los alumnos, durante la clase para aplicar los conocimientos del tema visto y al final para reforzarlos. Estamos convencidas de que las técnicas que abordaremos en el taller, utilizadas convenientemente, permitirán a los profesores conseguir efectos sorprendentes, inexplicables o, incluso, milagrosos en el salón de clase. Esta es una invitación a que los colegas docentes fomenten que sus alumnos vean a las matemáticas con ilusión y sorpresa en vez de desilusión y desengaño
%%%%%%%%%%%%1785%%%%%%%%%%%
\subsection{\sffamily Curso Fracciones sin dolor {\footnotesize (, )}} \label{reg-1785} \index{Rodríguez Carmona Hugo}
\noindent {\bfseries Hugo  Rodríguez Carmona}, {\tt hugo.rodriguezc@gmail.com}  {\slshape (El proyecto ¡Matemática sin dolor!)}\\
          \noindent Con la intención de remarcar la importancia del lenguaje para la comprensión de la matemática, se propone el siguiente curso en el cual se pretende indagar y reflexionar sobre  estrategias que faciliten la enseñanza de la división, adición y sustracción de quebrados así como la multiplicación y la división de quebrados, las operaciones básicas con números decimales, cómo calcular porcentajes y cómo hacer conversiones entre quebrados, decimales y porcentajes.

Creemos que al reconocer para qué sirven las operaciones aritméticas y cómo emplearlas,  los estudiantes entiendan los procesos de división, adición y sustracción de quebrados y al menos dos algoritmos para adicionar y sustraer quebrados para resolver problemas.

El alcance de este curso es Proporcionar alternativas para que los profesores ayuden a elevar la autoestima de sus estudiantes través de un enfoque amable, significativo y divertido de la matemática y poner al alcance de los padres de familia, estrategias efectivas que les permitan hacerles ver a sus hijos, que la matemática no es una materia aburrida y reservada “sólo para genios”. 
%%%%%%%%%%%%1428%%%%%%%%%%%
\subsection{\sffamily Curso: Taller de Resolución de Problemas-Intermedio {\footnotesize (, )}} \label{reg-1428} \index{Méndez Egbert}
\noindent {\bfseries Egbert  Méndez}, {\tt egbertmdz@ciencias.unam.mx}  {\slshape (Facultad de Ciencias, Universidad Nacional Autónoma de México (UNAM))}\\
          \noindent Curso: Taller de Resolución de Problemas. Tema: Actualización de profesores en contenidos matemáticos para el nivel primaria. Nivel: Tercero y cuarto de primaria Autores: Manuel López Mateos manuel@unam.mx, Jonathan Delgado Solórzano jonh2mx@gmail.com, Luis Ángel Jactthar Cruz luis_jac@hotmail.com, Angélica Cristina Velasquillo Ramírez symha_angie@hotmail.com, Jorge Alberto Ruíz Huerta seminario.jorge.1@hotmail.com, Damaso Ricardo Berriozábal Montiel r_berriozabañ@hotmail.com, Egbert Méndez Serrano egbertmdz@ciencias.unam.mx Justificación Dada la problemática que se presenta en la capacitación en matemáticas de los maestros de educación básica, tanto en su formación como en su actualización, el Seminario sobre la Enseñanza de las Matemáticas, que dirige el profesor Manuel López Mateos, en la Facultad de Ciencias de la UNAM, ha elaborado cursos para esta capacitación. Nuestras investigaciones nos han indicado que una de las deficiencias en dicha problemática radica en que la formación de maestros se centra en la didáctica de las matemáticas, soslayando los contenidos académicos. Recientemente, hemos notado que esta deficiencia no es única para México, sino se aplica a varios países de América Latina: Argentina, El Salvador, Nicaragua y Venezuela. Esta deficiencia formativa, se manifiesta en la práctica docente del día a día. Desde nuestro punto de vista, lejos de debatir cuestiones burocráticas, administrativas y políticas de educación, es necesario dedicar esfuerzos a esta situación de manera similar a como se destinan recursos y esfuerzos para abordar las diversas metodologías de la enseñanza en las matemáticas. Propuesta Desde que notamos esta situación, primero para la Ciudad de México, luego para el país y ahora para Latinoamérica, además de señalarla, venimos realizando cursos donde mostramos como creemos debe ser esta actualización, tanto de maestros en activo como de estudiantes prospectos a serlo. En estos cursos nos centramos en los contenidos académicos, sin intervenir en los aspectos didácticos. Tomamos esta posición pues creemos que la actualización de maestros se debe dar entre profesionales de las matemáticas (los matemáticos) y los profesionales de la educación (los maestros de educación básica), respetando, de esta manera, ambos aspectos de la enseñanza: la didáctica y el contenido. La forma natural para apuntalar los contenidos, es mediante la resolución de problemas. Por ello, nuestra forma de trabajo es mediante la resolución de problemas A pesar de que nuestra intención es abordar contenidos matemáticas exclusivamente, hemos desarrollado una pedagogía intuitiva potencialmente incipiente en quien estudia matemáticas. Por ello hemos propuesto que sean estudiantes de nivel avanzado y profesores de matemáticas de las universidades del país, quienes lleven a cabo esta misión. Cabe mencionar que el Seminario propone un curso global de actualización que debe ser impartido a todos los profesores de educación primaria, sin embargo, dadas las políticas educativas actuales, vemos poco viable la aplicación de esta opción. Por ello, hemos diseñado una alternativa, la cual radica en la impartición de cursos por nivel correspondientes a los grados educativos de educación primaria: Nivel A, que comprende la actualización en matemáticas de 10 y 20; Nivel B, que comprende la actualización en matemáticas de 30 y 40; y Nivel C, que comprende la actualización en matemáticas de 50 y 60. Con esto aseguramos que los profesores en activo que tomen uno de los cursos, tendrán una mejor preparación para el grado que imparten. El temario de los cursos fue elaborado, primero, a partir de un mapeo entre los contenidos académicos expuestos en los libros de texto gratuitos, o en documentos oficiales que presentan aprendizajes esperados de los países arriba mencionados, con el libro: Matemáticas, un enfoque de resolución de problemas para maestros de educación básica. López Mateos Editores, 2012. ISBN 978-607-95583-2-1 (en libro electrónico); luego, reordenamos los temas obtenidos de Un enfoque de solución… en base a su recurrencia y así conformar los temas de cada curso. En este curso, hacemos una extracción de los problemas correspondientes al Nivel B, lo cual nos permitirá mostrar nuestra mecánica de trabajo, así como abordar algunos contenidos matemáticos correspondientes con este nivel.
%%%%%%%%%%%%1695%%%%%%%%%%%
\subsection{\sffamily Experimentos Demostrativos en el Área de Divulgación de las Matemáticas {\footnotesize (, )}} \label{reg-1695} \index{Rosas Jaquelina Flores}
\noindent {\bfseries Jaquelina Flores Rosas}, {\tt ahome_jake@hotmail.com}  {\slshape (Benemérita Universidad Autónoma de Puebla (BUAP))}\\
          \noindent El aprendizaje, sobre todo en de las matemáticas, es uno de los problemas más apremiantes en los niveles básicos de la educación en nuestro país. Por tal razón, se han buscado métodos para que su aprendizaje sea más accesible para el estudiante. Específicamente, en la escuela primaria, el aprendizaje de las matemáticas es un problema, puesto que  muchas veces el profesor y el estudiante se enfrentan a un tema que les parece complicado y por lo generar aburrido. Por tal motivo, todo intento para introducir al estudiante en el aprendizaje de las matemáticas, sin que le resulte una actividad desagradable, pasa por el hecho de despertar sus inquietudes y su curiosidad. En este trabajo, reportamos nuestras experiencias al interactuar con cientos de escolares de nivel primaria, realizando actividades relacionadas con las matemáticas. Con base en estas experiencias podemos decir  que los escolares de nivel básico se sienten atraídos por aprender temas de matemáticas sin que les resulte difícil y aburrido.
%%%%%%%%%%%%811%%%%%%%%%%%
\subsection{\sffamily Historietas matemáticas de palma, una actividad para la inclusión {\footnotesize (, )}} \label{reg-811} \index{Fajardo Araujo María Carmen}
\noindent {\bfseries María Carmen Fajardo Araujo}, {\tt carmulita_@hotmail.com}  {\slshape (Universidad Autónoma de Coahuila, Facultad de Ciencias Físico-Matemáticas (UA de C))}\\
\noindent {\it  Coautor: Luz María Jiménez Juárez        }\\
\noindent La educación amplía las oportunidades, así como las relaciones interculturales permitiendo reducir las desigualdades sociales y culturales, la tarea del docente tiene que favorecer entre los estudiantes el reconocimiento a la pluralidad social y cultural, donde la escuela sea un espacio en el que se promueva, practique y aprecie esa diversidad como cotidiana en la vida del alumno. Los alumnos que asisten a la secundaria de Tancoyol pertenecen a comunidades con presencia “Pame”, de ahí la razón para buscar una actividad que favoreciera el reconocimiento de este grupo, sus características, costumbres, tradiciones, etc., entre la comunidad estudiantil de la secundaria.
%%%%%%%%%%%%1582%%%%%%%%%%%
\subsection{\sffamily El Rueda-metro, el Hipsómetro y la Bazuca, tres instrumentos para trabajar en Geometría. {\footnotesize (, )}} \label{reg-1582} \index{Macías Romero Juan Carlos}
\noindent {\bfseries Juan Carlos Macías Romero}, {\tt jcmacias70@hotmail.com}  {\slshape (Secretaría de Educación Pública del Estado de Puebla (SEP))}\\
\noindent {\it  Coautores: Beatriz  Moreno Tochihuitl, Carmina  Jiménez Flores      }\\
\noindent En esta plática mencionaremos cómo se puede construir el rueda-metro, el Hipsómetro y la Bazuca en el aula de matemáticas. Este material es de gran utilidad en la clase de geometría, desde su diseño hasta su aplicación. Hemos comprobado que el aprendizaje de los estudiantes ha sido más significativo al usar estos instrumentos. Con estos aparatos se pueden diseñar clases innovadoras para lograr un aprendizaje más efectivo, para desarrollar las competencias de los estudiantes y para trabajar con otros temas de matemáticas (ecuaciones, cálculo de áreas, manejo e interpretación de la información, proporcionalidad, etc.).
%%%%%%%%%%%%1824%%%%%%%%%%%
\subsection{\sffamily Experiencias educativas {\footnotesize (, )}} \label{reg-1824} \index{Varela Molinar Javier Saúl}
\noindent {\bfseries Javier Saúl Varela Molinar}, {\tt javiersvarela@hotmail.com}  {\slshape (ESCUELA SECUNDARIA TÉCNICA N° 6 (EST 6) PROFESOR DE MATEMATICAS EN 1°, 2° Y 3° GRADO)}\\
          \noindent Esta actividad la he desarrollado en dos momentos. El primero momento, a mediados del mes de Enero, mientras se abordaba el contenido 9.3.3. Resolución de problemas geométricos mediante el teorema de Tales; el segundo momento es a finales del mes de Abril, mientras se aborda el contenido 9.4.5 Explicitación y uso de las razones trigonométricas seno, coseno y tangente.   La actividad consiste en lo siguiente. Se pide a los alumnos se integren en equipos de tres personas, y se hace el siguiente cuestionamiento: “Queremos comprobar que la altura de la canasta de basquetbol de nuestra escuela cumple con la medida reglamentaria de esta disciplina. ¿de qué manera podemos medir la altura si nuestra única herramienta disponible es un metro de madera?” Para poder realizar esta actividad, primero se hacen las formulaciones de las hipótesis de medición en el interior del aula, en donde se escuchan y se anotan las propuestas hechas por los jóvenes. Se da oportunidad de escuchar a todos los equipos y se pie que justifiquen sus propuestas. Hay opiniones de todo tipo, desde treparse a la canasta y a partir de ahí amarrar los cinturones de los integrantes para checar la altura; otros más dicen que haciendo una columna humana para comprobar la altura. Y hay quien dice que midiendo la sombra que proyecta para calcular la altura. En ésta última propuesta es en donde se hace hincapié de utilizar los procedimientos matemáticos con los que se cuentan hasta este momento, dejando un registro para poder realizar la comprobación correspondiente, así que armados con el metro de madera, se procede a medir la sombra que la canasta proyecta en el piso.
%%%%%%%%%%%%1887%%%%%%%%%%%
\subsection{\sffamily La elaboración y el uso de materiales didácticos como una estrategia en la enseñanza de las fracciones en la escuela primaria {\footnotesize (, )}} \label{reg-1887} \index{Clemente Lara San Juana}
\noindent {\bfseries San Juana  Clemente Lara}, {\tt playalejana@hotmail.com}  {\slshape ()}\\
\noindent {\it  Coautor: María Luisa Gama Evans        }\\
\noindent El identificar el desarrollo de habilidades matemáticas a través del diseño de materiales, es en la actualidad una necesidad que el docente enfrenta día a día como un reto a su labor, en la búsqueda incesante de aprendizajes significativos. En la medida en que el docente haga uso de su creatividad, infundirá en sus alumnos,  la confianza al momento de adquirir los conocimientos en uno de los temas que presentan cierta dificultad tanto para su enseñanza como para el aprendizaje. La reflexión que el docente haga de la importancia del uso de materiales manipulables, es vital en esta actividad, porque utilizando el juego, será capaz de contar con los recursos necesarios para resolver problemas planteados mediante diseño de situaciones de aprendizaje y consignas, que le auxiliarán a identificar en sus alumnos las habilidades matemáticas tan necesarias en el desarrollo educacional de los jóvenes hoy en día. El taller se desarrolla en un lapso de 4 a 6 hrs. hrs., en donde los docentes elaborarán materiales manipulables e intercambiarán sus experiencias acerca de las diferentes formas de impartir este tema en sus instituciones.  
%%%%%%%%%%%%1882%%%%%%%%%%%
\subsection{\sffamily A la misma distancia {\footnotesize (, )}} \label{reg-1882} \index{Quezada Muñoz Javier}
\noindent {\bfseries Javier  Quezada Muñoz}, {\tt quezada.jav@hotmail.com}  {\slshape ()}\\
          \noindent La presente Estrategia de Aprendizaje, fue producto del trabajo realizado por un grupo de docentes durante el “Primer Seminario de Profesionalización para Profesores sobre Experiencias de Aprendizaje en el Aula”, el cual se realizó en la Ciudad de México en diciembre de 2011, dicha Estrategia pertenece a los contenidos a desarrollar para el primer grado de educación secundaria, (séptimo grado de Educación Básica, Reforma 2011), durante el primer bimestre. Como la estrategia fue elaborada en el mes de diciembre de 2011, esta fue desarrollada durante el tercer bimestre del ciclo escolar 2011 – 2012 con el grupo de primer grado, sección “E”, de la Escuela Secundaria General número 49 “María E. Villarreal Cavazos”, perteneciente a la zona # 27, en la ciudad de Gral. Escobedo, N. L.
%%%%%%%%%%%%1881%%%%%%%%%%%
\subsection{\sffamily La evaluación por criterios. Participación activa del alumno {\footnotesize (, )}} \label{reg-1881} \index{Guevara Araiza Albertico}
\noindent {\bfseries Albertico  Guevara Araiza}, {\tt alberticoguevara70@gmail.com}  {\slshape ()}\\
          \noindent La evaluación externa hacia las instituciones educativas de la educación básica en México es un proceso que se instauró definitivamente desde 2006. Tanto las escuelas primarias como secundarias y las de educación media superior, así como los docentes, padres de familia y alumnos deberán aprender a aprovechar esta realidad educativa: la utilización de sus resultados para convertirla en información en beneficio de los alumnos es la medida más inteligente. Es en este contexto que se plantea la presente propuesta de intervención didáctica, como una de las formas que el docente puede utilizar para generar formas y formatos de evaluación que contemplen al alumno como ser holístico y dinámico.
%%%%%%%%%%%%1884%%%%%%%%%%%
\subsection{\sffamily Estrategia para el desarrollo de la Competencia argumentativa en alumnos de primer grado de matemáticas en Secundaria {\footnotesize (, )}} \label{reg-1884} \index{Solórzano Torres María Eugenia}
\noindent {\bfseries María Eugenia  Solórzano Torres}, {\tt fachust@hotmail.com}  {\slshape ()}\\
          \noindent Se describe un acercamiento hacia el desarrollo de la competencia de la ARGUMENTACIÓN en matemáticas a través de la Pedagogía Dialogante utilizando la EXPOSICIÓN de los alumnos como recurso para el aprendizaje de las matemáticas, y mediante una actividad titulada Flor Geométrica. Donde se trata de diseñar una Flor a partir de un área determinada y solo se pueden usar algunas figuras geométricas como: triángulos, cuadrados, rectángulos y trapecios. Se propone un diseño individual que corresponde al bosquejo y en equipo se decide cual será reproducida a escala en el papel bond y presentada al grupo cumpliendo las condiciones antes mencionadas.
%%%%%%%%%%%%1885%%%%%%%%%%%
\subsection{\sffamily Semejanza {\footnotesize (, )}} \label{reg-1885} \index{Guadarrama Fuentes José Guadalupe}
\noindent {\bfseries José Guadalupe Guadarrama Fuentes}, {\tt joseguadarrama63@hotmail.com}  {\slshape ()}\\
          \noindent Se trabaja una situación haciendo uso de las propiedades de semejanza en triángulos rectángulos y relaciones de proporcionalidad: La razón como cociente de dos cantidades en el cálculo de la altura de un árbol, poste, edificio, etc. Se pretende que los estudiantes resuelvan problemas que impliquen reconocer, estimar y medir ángulos, utilizando el grado como unidad de medida. (primer bloque), así como también problemas de comparación de razones, con base en la noción de equivalencia. (segundo bloque)Determinar los criterios de congruencia de triángulos a partir de construcciones con información determinada. (cuarto bloque).
%%%%%%%%%%%%1442%%%%%%%%%%%
\subsection{\sffamily Construcción del omnipoliedro utilizando la papiroflexia (origami) {\footnotesize (, )}} \label{reg-1442} \index{Tovar Monsiváis María Ofelia}
\noindent {\bfseries María Ofelia Tovar Monsiváis}, {\tt ofelia_tovar_21@hotmail.com}  {\slshape ( Universidad Autónoma de San Luis Potosí (UASLP)  Facultad de Ciencias )}\\
          \noindent Un omnipoliedro es una estructura realizada con los armazones de los cinco poliedros regulares (sólidos platónicos) de manera tal que se encuentran inscritos uno dentro del otro, llevándolo a cabo mediante el arte de doblar papel llamado origami (modular). El objetivo del taller es transmitir los conceptos que se engloban (origami, papiroflexia, poliedros, poliedros regular, vértice, cara, arista, etc.),  mediante cuatro diferentes técnicas de armado (módulo), aplicándolos para la construcción de estos poliedros de esta manera se medirá el grado de coordinación entre lo real y lo abstracto, se visualizará la comprensión de la geometría, ayudando al desarrollo de la destreza, exactitud y precisión manual. En cada sesión, se explicará la parte teórica como el módulo a realizar. Es recomendable que los participantes asistan a todas las sesiones, debido que es un trabajo consecutivo además de realizar lo más exacto posible los dobleces para su óptimo ensamble.
%%%%%%%%%%%%1019%%%%%%%%%%%
\subsection{\sffamily Demostraciones de conceptos geométricos a partir del doblado de papel {\footnotesize (, )}} \label{reg-1019} \index{RIVERA BOBADILLA OLGA}
\noindent {\bfseries OLGA  RIVERA BOBADILLA}, {\tt orb@uaemex.mx}  {\slshape (Facultad de Ciencias, Universidad Autónoma del Estado de México, (UAEM))}\\
\noindent {\it  Coautor: BENITO FERNANDO MARTÍNEZ SALGADO        }\\
\noindent El uso de doblado de papel nos permite hacer una transición entre la experiencia de una actividad lúdica y la abstracción de conceptos geométricos, estableceremos una relación entre actividades y demostraciones formales de: construcciones de triángulos, teorema de Pitágoras, cónicas y la trisección de un ángulo.
%%%%%%%%%%%%347%%%%%%%%%%%
\subsection{\sffamily Geometría  Diferencial en el espacio de Minkowski de dimensión tres {\footnotesize (, )}} \label{reg-347} \index{Ruiz Hernández Gabriel}
\noindent {\bfseries Gabriel  Ruiz Hernández}, {\tt gruiz@matem.unam.mx}  {\slshape (Instituto de Matemáticas de la UNAM)}\\
          \noindent El objetivo del minicurso es dar una introducción a la Geometría Diferencial en el espacio de Minkowski de dimensión tres. Estas son parte de la matemáticas que uso Albert Einstein para desarrollar la teoría de la Relatividad Especial que presento en 1905. Así que puede verse como un minicurso de aspectos matemáticos de dicha teoria.Esta dirigido a estudiantes de matemáticas y áreas afines que hayan cursado cinco semestres de su carrera. Se dejaran algunos ejercicios para que el estudiante pueda tener practica con las técnicas que se van a explicar en clase.
%%%%%%%%%%%%1615%%%%%%%%%%%
\subsection{\sffamily Como abordar problemas de preparación para olimpiada de matemáticas  y mostrar diversas soluciones {\footnotesize (CC, Bach)}} \label{reg-1615} \index{Briones Pérez Fredy}
\noindent {\bfseries Fredy  Briones Pérez}, {\tt fimpara_06@hotmail.com}  {\slshape (Universidad Autónoma de Tlaxcala (UAT))}\\
\noindent {\it  Coautores: María de la Paz Pérez Reyes, María del Carmen Sampedro Portillo      }\\
\noindent Este curso nos brinda la oportunidad de reflexionar y resolver problemas para olimpiada de matemáticas, en especial del área de geometría y de otras áreas, y así poder enseñar al alumno diferentes formas para resolver un mismo problema, con herramientas básicas.  (Se desarrollara la teoría para poder resolver algunos de estos problemas.).  Los problemas a desarrollar son propuestos por el  CIMAT. Con esto se pretende que los  profesores puedan capacitar a alumnos de una manera no tan rígida y con una visión más amplia de lo que es resolver esta clase de problemas.
%%%%%%%%%%%%517%%%%%%%%%%%
\subsection{\sffamily La lógica y el juego {\footnotesize (, )}} \label{reg-517} \index{Góngora Vega Luis Ceferino}
\noindent {\bfseries Luis Ceferino Góngora Vega}, {\tt luiscef@yahoo.com.mx}  {\slshape (Escuela Secundaria Estatal N° 13 Lic. Rafael Matos Escobedo Escuela Preparatoria Oxkutzcab)}\\
          \noindent Durante nuestro proceso de vida como estudiantes aprendemos de muchas formas y ahora como facilitadores del aprendizaje lo más sencillo es enseñar de la forma en que aprendimos. El presente taller pretende acercarse a uno de los fundamentos de la matemática; el desarrollo del pensamiento lógico de los alumnos, ofreciendo algunas estrategias prácticas para contribuir al desarrollo de sus habilidades de manera entretenida y productiva. Este trabajo se realizó con estudiantes de 2° grado del tercer semestre y que estaban inscritos en las secciones A, B y C de la Escuela Preparatoria “Oxkutzcab” de la ciudad de Oxkutzcab, Yucatán, México.
%%%%%%%%%%%%1592%%%%%%%%%%%
\subsection{\sffamily ¿La radicación y la potenciación como inversas? El caso de la raíz cuadrada {\footnotesize (, )}} \label{reg-1592} \index{Colín Uribe María Patricia}
\noindent {\bfseries María Patricia Colín Uribe}, {\tt patricia_c_u@hotmail.com}  {\slshape (INSTITUTO POLITECNICO NACIONAL)}\\
\noindent {\it  Coautor: Gustavo  Martínez Sierra        }\\
\noindent Este trabajo es la continuación de investigaciones de Lorenzo, D. (2005) y Colín, M. (2006), las cuales muestran las concepciones que estudiantes de nivel básico hasta nivel superior tienen acerca de la operación raíz cuadrada, evidenciando las “disfunciones escolares” que este operador presenta en el tránsito del contexto aritmético al algebraico y del algebraico al funcional. Uno de estos resultados muestra que los estudiantes de estos niveles miran a las operaciones de potenciación y radicación como inversas, en particular a la raíz cuadrada como inversa de elevar al cuadrado. Este trabajo pretende que, a través de una secuencia de actividades, estudiantes de nivel bachillerato concluyan que estas operaciones sólo son inversas bajo ciertas condiciones. . Palabras clave: raíz cuadrada, operación inversa, potenciación, radicación, disfunción
%%%%%%%%%%%%956%%%%%%%%%%%
\subsection{\sffamily La Parábola:  Una aplicación con enfoque en Aprendizaje Basado en Problemas (ABP) y Geometría Dinámica. {\footnotesize (, )}} \label{reg-956} \index{Martínez Medina Jonathan Enrique}
\noindent {\bfseries Jonathan Enrique Martínez Medina}, {\tt jona_martinez7@hotmail.com}  {\slshape (Universidad Autónoma de San Luis Potosí (UASLP))}\\
\noindent {\it  Coautor: Yessica  Miranda Pineda        }\\
\noindent A través de los años se ha llegado a la conclusión de que el docente no está preparado en cuanto a la fundamentación teórica que utiliza y por consecuencia el aprendizaje del alumno no es el deseado. Por una parte el docente sigue dado sus clases así como lo señala la teoría conductista, y nosotros como matemáticos educativos tenemos como objetivo cambiar esta forma tradicional de enseñanza y diseñar, planear, ejecutar y evaluar procesos de enseñanza-aprendizaje óptimos para satisfacer las necesidades del contexto aúlico. Expondremos cómo el método “Aprendizaje basado en problemas” aplicado al tema de parábola, puede ayudar. Consideramos importante para ello, que el docente problematice el saber. Partimos de sugerir al alumno que investigue  sobre por lo menos una aplicación de la parábola en la vida real; sobre la cual se propone ir problematizando al momento que vaya realizando la actividad, con la orientación del profesor. Se sugiere también que se apoye en el software Geogebra, con el propósito de que el alumno utilice la geometría dinámica para adquirir y afianzar las propiedades de la parábola.
%%%%%%%%%%%%1587%%%%%%%%%%%
\subsection{\sffamily ¿Qué efectos causa en los estudiantes la incorporación de gráficas en el tratamiento de algunos conceptos del Cálculo Diferencial? Una experiencia  en el Nivel Medio Superior {\footnotesize (, )}} \label{reg-1587} \index{Colín Uribe María Patricia}
\noindent {\bfseries María Patricia Colín Uribe}, {\tt patricia_c_u@hotmail.com}  {\slshape (INSTITUTO POLITECNICO NACIONAL)}\\
\noindent {\it  Coautores: Celia Araceli Islas Salomón, Fernando  Morales Téllez      }\\
\noindent En el Nivel Medio Superior (NMS), la enseñanza del Cálculo tiende a sobrevalorar los procedimientos analíticos y la algoritmización, dejando de lado a los argumentos visuales por no considerarlos como puramente “matemáticos”, a pesar de que la visualización es una habilidad en los seres humanos que, por naturaleza siempre será empleada para representar una parte o la aproximación de la realidad que se estudia. Algunas investigaciones en el campo de la Educación Matemática, demuestran que cuando un estudiante logra incorporar elementos visuales como parte de su actividad matemática, podrá transitar entre las diversas representaciones: algebraicas, geométricas, numéricas y verbales y de esta forma, darle un significado mas rico al saber en cuestión. La investigación en el área de la enseñanza de las matemáticas nos permite identificar fenómenos sobre ciertos aspectos del proceso enseñanza-aprendizaje en esta materia. Así, de los resultados obtenidos, podemos proponer secuencias de aprendizaje que permitan al estudiante tener mejores resultados en la adquisición de habilidades y conceptos matemáticos. Esta investigación se realizó en el Centro de Estudios Científicos y Tecnológicos “Narciso Bassols” del Instituto Politécnico Nacional y tiene como objetivo, mostrar las respuestas de un grupo de estudiantes que llevó un curso de Cálculo Diferencial basado en elementos visuales, y un grupo de estudiantes que sólo recibieron instrucción meramente analítica, después de haber trabajado con una secuencia de actividades gráficas. Nuestro interés se centrará a conceptos como límites, dominio y rango de un función, función par o impar, intervalos de crecimiento y decrecimiento de funciones con características especiales.   Palabras clave: Visualización, representación gráfica, proceso analítico, dominio, rango
%%%%%%%%%%%%1235%%%%%%%%%%%
\subsection{\sffamily ¿Qué se sentirá ser el profesor? {\footnotesize (, )}} \label{reg-1235} \index{Almazán Torres Elizabeth}
\noindent {\bfseries Elizabeth  Almazán Torres}, {\tt mateeli@yahoo.com.mx}  {\slshape (Universidad Autónoma del Estado de México (UAEMéx))}\\
\noindent {\it  Coautor: Rosario  Sánchez Pérez        }\\
\noindent Es una propuesta de dar una clase para involucrar a los estudiantes de tal manera que valoren, como cada docente se siente al no poder llevar a cabo el dar el conocimiento como uno lo plantea y que cada vez uno debe ajustarse a las necesidades que surgen en el momento y no estar necios en lo que se planea
%%%%%%%%%%%%1883%%%%%%%%%%%
\subsection{\sffamily La resignificación del uso de las gráficas a través de la modelación, la graficación y la tecnología {\footnotesize (, )}} \label{reg-1883} \index{Briceño Solís Eduardo}
\noindent {\bfseries Eduardo  Briceño Solís}, {\tt ebriceno@cinvestav.mx}  {\slshape (Centro de Investigación y de Estudios Avanzados del (IPN), Departamento de Matemática Educativa.)}\\
          \noindent Se reporta una experiencia de una situación compuesta de actividades de modelación y graficación en un curso con estudiantes y profesores de nivel bachillerato del estado de México. El propósito de la ponencia es mostrar una resignificación del uso de las gráficas  en una situación específica (Cordero, Cen & Suárez, 2010). Las actividades se centraron en el rol que desempeña, en la enseñanza y aprendizaje de las matemáticas, la modelación, graficación y el uso de tecnología escolar. Dichas actividades se sustenta por la Teoría Socioespistemología que postula que las gráficas son argumentativas del Cálculo (Cordero, 2008) y que la modelación es una construcción del conocimiento matemático en sí mismo (Cordero, 2006; Méndez, 2008 y Suárez y Cordero, 2010). Lo cual, conlleva a desarrollar estudios del uso y desarrollo de prácticas de graficación y modelación, que favorecen una matemática funcional en oposición a una utilitaria. Las actividades propuestas se sitúan en una situación de transformación, donde por medio del uso de la gráfica en calculadoras gráficas y sensores de movimiento, se dio significado a la función de la forma f(x) = Ax2+C para x>0. Sin embargo en el desarrollo de las actividades y reflexiones de las mismas, se reporta como en la organización de estudiantes y profesores se construye una resignificación  de los coeficientes A y C de la función anterior, por medio de comportamientos gráficos, el uso de la gráfica.
%%%%%%%%%%%%1176%%%%%%%%%%%
\subsection{\sffamily Qué tanto sirve la prueba EXANI I y II para elegir a nuestros aspirantes {\footnotesize (, )}} \label{reg-1176} \index{Gamón Madrid Araceli C.}
\noindent {\bfseries Araceli C. Gamón Madrid}, {\tt araceli.gamonm@gmail.com}  {\slshape (Universidad Autónoma de Zacatecas (UAZ))}\\
          \noindent  La Universidad Autónoma de Zacatecas, aplica la prueba EXANI I y II a los aspirantes para ingresar al nivel medio superior o al nivel superior, esta prueba refleja el resultado del nivel académico con el cual ingresa el alumno, pero al dar seguimiento al comportamiento del alumno en su área de desenvolvimiento se presentan problemas con los jóvenes que obtienen algunas veces muy buenos resultados, pues éstos desertan o cuentan un nivel de aprovechamiento es bajo.  
%%%%%%%%%%%%518%%%%%%%%%%%
\subsection{\sffamily Material interactivo en la enseñanza-aprendizaje de las operaciones algebraicas en secundaria {\footnotesize (, )}} \label{reg-518} \index{Góngora Vega Luis Ceferino}
\noindent {\bfseries Luis Ceferino Góngora Vega}, {\tt luiscef@yahoo.com.mx}  {\slshape (Escuela Secundaria Estatal N° 13 Lic. Rafael Matos Escobedo Escuela Preparatoria Oxkutzcab)}\\
          \noindent El propósito del presente fue diseñar un material interactivo educativo con miras al fortalecimiento de los procesos enseñanza y aprendizaje de las operaciones algebraicas suma, resta, multiplicación y división con alumnos del tercer grado de educación media básica de la Escuela Secundaria Estatal N°13 “Lic. Rafael Matos Escobedo” de la ciudad de Oxkutzcab, Yucatán, México durante el curso escolar 2009-2010.
%%%%%%%%%%%%1886%%%%%%%%%%%
\subsection{\sffamily 2.	Resolución de problemas pre-algebraicos a través de situaciones didácticas en telesecundaria {\footnotesize (, )}} \label{reg-1886} \index{Morales Morales Ana Josefina}
\noindent {\bfseries Ana Josefina Morales Morales}, {\tt ana_selarom@yahoo.com.mx}  {\slshape ()}\\
          \noindent En nuestra experiencia docente encontramos que los alumnos abandonan sus tareas de matemáticas, porque los ejercicios les hablan de hechos históricos o de géneros, familias o descubrimientos, que ellos ignoran o bien por un lenguaje demasiado técnico y por lo tanto incomprensible. Por lo que se diseñó un conjunto de Situaciones didácticas para reforzar el conocimiento matemático de los alumnos de segundo año de secundaria. Se eligió el tema Resolución de Problemas Pre-Algebraicos pues se considera que de su comprensión depende que el alumno pueda trabajar con otro tipo de expresiones algebraicas, los subtemas que se abarcan son: Operaciones de Números con signo, Reconocimiento de variables, Formular expresiones algebraicas, Reducción de términos semejantes. Para su estudio cada tema se presenta dentro de una situación didáctica, por medio de ella se espera que el alumno logre descubrir alguna aplicación práctica del subtema presentado al tiempo que fundamenta el porqué de las reglas formales utilizadas.
%%%%%%%%%%%%1368%%%%%%%%%%%
\subsection{\sffamily Imaginación espacial {\footnotesize (, )}} \label{reg-1368} \index{Orozco Vaca Luz Graciela}
\noindent {\bfseries Luz Graciela Orozco Vaca}, {\tt lorozco_@hotmail.com}  {\slshape (CENTRO DE INVESTIGACION Y ESTUDIOS AVANZADOS DEL POLITECNICO NACIONAL (CINVESTAV) MATEMÁTICA EDUCATIVA)}\\
\noindent {\it  Coautor: Ricardo  Quintero Zazueta        }\\
\noindent El propósito de este taller es promover la reflexión de los profesores de secundaria alrededor de habilidades espaciales que pueden ser base para el posterior estudio de la geometría. Se desarrollaran, en dos partes, actividades con figuras tridimensionales. La primera parte consistirá en llevar a cabo tareas utilizando el Cubo de Soma mediante hojas de trabajo y tendrá duración de 5 horas. En la segunda parte se dará una pequeña exposición teórica del tema y se recibirán comentarios de los participantes.
%%%%%%%%%%%%882%%%%%%%%%%%
\subsection{\sffamily Los poliedros regulares, estudio y aplicación mediante papiroflexia {\footnotesize (, )}} \label{reg-882} \index{Rojas Monroy María del Rocío}
\noindent {\bfseries María del Rocío  Rojas Monroy}, {\tt mrrm@uaemex.mx}  {\slshape (FACULTAD DE CIENCIAS, UNIVERSIDAD AUTÓNOMA DEL ESTADO DE MÉXICO (UAEM))}\\
\noindent {\it  Coautores: Olga  Rivera Bobadilla, Enrique  Casas Bautista, Alejandro  Contreras Balbuena    }\\
\noindent En este taller se elaborar\'an objetos basados en los poliedros regulares, y se har\'a \'énfasis en que estos objetos pueden usarse para construir objetos m\'as complejos como pueden ser fractales. En este proceso se utilizar\'a papiroflexia modular adem\'as de conceptos de Teor\'ia de Gr\'aficas tales como coloraci\'on y planaridad. Dadas las carácter\'isticas propias de esta actividad y para promover un aprendizaje significativo as\'i como  diversificar la cantidad de ejemplos se propone trabajar en equipos peque\~nos, cada uno con un instructor.
%%%%%%%%%%%%599%%%%%%%%%%%
\subsection{\sffamily Curso: Taller de Resolución de Problemas-Básico {\footnotesize (, )}} \label{reg-599} \index{Méndez Egbert}
\noindent {\bfseries Egbert  Méndez}, {\tt egbertmdz@ciencias.unam.mx}  {\slshape (Facultad de Ciencias, Universidad Nacional Autónoma de México (UNAM))}\\
          \noindent Curso: Taller de Resolución de Problemas. Tema: Actualización de profesores en contenidos matemáticos para el nivel primaria. Nivel: Primero y segundo de primaria Autores: Manuel López Mateos manuel@unam.mx, Jonathan Delgado Solórzano jonh2mx@gmail.com, Luis Ángel Jactthar Cruz luis_jac@hotmail.com, Angélica Cristina Velasquillo Ramírez symha_angie@hotmail.com, Jorge Alberto Ruíz Huerta seminario.jorge.1@hotmail.com, Damaso Ricardo Berriozábal Montiel r_berriozabañ@hotmail.com, Egbert Méndez Serrano egbertmdz@ciencias.unam.mx. Justificación Dada la problemática que se presenta en la capacitación en matemáticas de los maestros de educación básica, tanto en su formación como en su actualización, el Seminario sobre la Enseñanza de las Matemáticas, que dirige el profesor Manuel López Mateos, en la Facultad de Ciencias de la UNAM, ha elaborado cursos para esta capacitación. Nuestras investigaciones nos han indicado que una de las deficiencias en dicha problemática radica en que la formación de maestros se centra en la didáctica de las matemáticas, soslayando los contenidos académicos. Recientemente, hemos notado que esta deficiencia no es única para México, sino se aplica a varios países de América Latina: Argentina, El Salvador, Nicaragua y Venezuela. Esta deficiencia formativa, se manifiesta en la práctica docente del día a día. Desde nuestro punto de vista, lejos de debatir cuestiones burocráticas, administrativas y políticas de educación, es necesario dedicar esfuerzos a esta situación de manera similar a como se destinan recursos y esfuerzos para abordar las diversas metodologías de la enseñanza en las matemáticas. Propuesta Desde que notamos esta situación, primero para la Ciudad de México, luego para el país y ahora para Latinoamérica, además de señalarla, venimos realizando cursos donde mostramos como creemos debe ser esta actualización, tanto de maestros en activo como de estudiantes prospectos a serlo. En estos cursos nos centramos en los contenidos académicos, sin intervenir en los aspectos didácticos. Tomamos esta posición pues creemos que la actualización de maestros se debe dar entre profesionales de las matemáticas (los matemáticos) y los profesionales de la educación (los maestros de educación básica), respetando, de esta manera, ambos aspectos de la enseñanza: la didáctica y el contenido. La forma natural para apuntalar los contenidos, es mediante la resolución de problemas. Por ello, nuestra forma de trabajo es mediante la resolución de problemas A pesar de que nuestra intención es abordar contenidos matemáticas exclusivamente, hemos desarrollado una pedagogía intuitiva potencialmente incipiente en quien estudia matemáticas. Por ello hemos propuesto que sean estudiantes de nivel avanzado y profesores de matemáticas de las universidades del país, quienes lleven a cabo esta misión. Cabe mencionar que el Seminario propone un curso global de actualización que debe ser impartido a todos los profesores de educación primaria, sin embargo, dadas las políticas educativas actuales, vemos poco viable la aplicación de esta opción. Por ello, hemos diseñado una alternativa, la cual radica en la impartición de cursos por nivel correspondientes a los grados educativos de educación primaria: Nivel A, que comprende la actualización en matemáticas de 10 y 20; Nivel B, que comprende la actualización en matemáticas de 30 y 40; y Nivel C, que comprende la actualización en matemáticas de 50 y 60. Con esto aseguramos que los profesores en activo que tomen uno de los cursos, tendrán una mejor preparación para el grado que imparten. El temario de los cursos fue elaborado, primero, a partir de un mapeo entre los contenidos académicos expuestos en los libros de texto gratuitos, o en documentos oficiales que presentan aprendizajes esperados de los países arriba mencionados, con el libro: Matemáticas, un enfoque de resolución de problemas para maestros de educación básica. López Mateos Editores, 2012. ISBN 978-607-95583-2-1 (en libro electrónico); luego, reordenamos los temas obtenidos de Un enfoque de solución… en base a su recurrencia y así conformar los temas de cada curso. En este curso, hacemos una extracción de los problemas correspondientes al Nivel A, lo cual nos permitirá mostrar nuestra mecánica de trabajo, así como abordar algunos contenidos matemáticos correspondientes con este nivel.
%%%%%%%%%%%%1758%%%%%%%%%%%
\subsection{\sffamily Aplicación de Producto Punto y Vectorial en un CAD {\footnotesize (CC, 2Lic)}} \label{reg-1758} \index{Águeda Herrera Mario Enrique}
\noindent {\bfseries Mario Enrique Águeda Herrera}, {\tt dsc@itcomitan.edu.mx}  {\slshape (Departamento de Sistemas y Computación.  Instituto Tecnológico de Comitán (ITC))}\\
          \noindent Una de las transformadas geométricas útiles en el diseño por computadora, es la rotación de figuras, ya sea con pivote en el origen o de cualquier punto. Para realizar la transformada se necesita el valor del ángulo de la rotación. Concretamente se necesitan conocer el coseno y el seno de dicho ángulo. En la práctica, el software para Diseño Asistido por Computadora (CAD) debe permitir al usuario realizar la rotación utilizando el ratón (mouse) de la computadora. Mediante los eventos ocurridos cuando se presiona el mouse (mousepressed) y cuando se suleta (mousereleased) se puede obtener un vector inicial y un vector final de la rotación. Utilizando el  producto punto de vectores, \overrightarrow{V1} \cdot \overrightarrow{V2} = \left | \overrightarrow{V1} \right | \cdot \left | \overrightarrow{V1} \right | \cos (\Theta ), se obtiene el coseno del angulo \Theta el cual indica los grados que se rota la figura. Así también con el producto Vectorial (cruz),  \overrightarrow{V1} \times  \overrightarrow{V2} = \left | \overrightarrow{V1} \right | \cdot \left | \overrightarrow{V1} \right | \sin (\Theta )  \widehat{i}, se obtiene el seno del ángulo de rotación. Se mostrará que usar estas operaciones vectoriales se optimiza el código de rotación, pues no se usarán las librerías matemáticas del lenguaje y las coordenadas transformadas se obtienen mediante operaciones básicas de aritmética. Utilizando el lenguaje de Programación Java, los participantes desarrollarán su propio CAD donde realizarán las operaciones básicas de transformación de figuras incluyendo la metodología que aquí se presenta.  
%%%%%%%%%%%%1555%%%%%%%%%%%
\subsection{\sffamily Hablemos el mismo idioma: ¿Y a ti cómo te hablan las matemáticas? {\footnotesize (, )}} \label{reg-1555} \index{Gómez Leal Diana Sarait}
\noindent {\bfseries Diana Sarait Gómez Leal}, {\tt dizzy16_1@hotmail.com}  {\slshape (Como estudiante; Universidad Autónoma de San Luís Potosí (UASLP), Facultad de Ciencias Como profesor; Instituto Tecnológico de San Luís Potosí (ITSLP). Departamento de Ciencias Básicas )}\\
\noindent {\it  Coautores: Eustorgia  Puebla Sánchez, Julio Heriberto Mata Salazar      }\\
\noindent “¿Cómo mejorar el manejo del lenguaje algebraico, en el aula para estudiantes de nuevo ingreso del Instituto Tecnológico de San Luis Potosí?”objetivo: Equilibrar el manejo del lenguaje algebraico en el aula para estudiantes de nuevo ingreso del INSTITUTO TECNOLÓGICO DE SAN LUIS POTOSÍ, para mejorar la asimilación de nuevos conceptos matemáticos en cursos posteriores. Pretendemos que en el curso propedéutico del semestre agosto – diciembre de 2012, el profesor utilice un lenguaje algebraico distinto al habitual, de tal manera que cada símbolo represente una situación diferente, haciendo que no sea un trabajo solamente expositivo para lograr solucionar algún problema, sino que el alumno pueda llegar a una etapa de comprensión más madura, del lenguaje algebraico.
%%%%%%%%%%%%1802%%%%%%%%%%%
\subsection{\sffamily El uso de artizones como un medio para lograr aprendizaje de matemáticas, entre los alumnos de primer cuatrimestre de la Universidad Tecnológica de Aguascalientes {\footnotesize (, )}} \label{reg-1802} \index{González Ramírez Mónica}
\noindent {\bfseries Mónica  González Ramírez}, {\tt mgonzalez@utags.edu.mx}  {\slshape (Coordinación de matemáticas Universidad Tecnológica de Aguascalientes (UTA))}\\
          \noindent El nivel de competencia matemática de los alumnos es insuficiente en conocimientos y habilidades para su desempeño en el ámbito de la educación superior, ya que tienen dificultades al aplicar las matemáticas para resolver problemas de su especialidad; analizar de una manera crítica la información; utilizar técnicas e instrumentos matemáticos para modelar y tomar decisiones; así como para saber calcular, representar, comunicar, argumentar información técnica y toma de decisiones. El desarrollo e implementación de nuevas técnicas “Artizones” dentro del marco de una teoría constructivista, contribuye al conocimiento de nuevos resultados, que amplían el campo de aplicabilidad en la enseñanza y aprendizaje de los conceptos básicos de matemáticas.
En la presente investigación se hará referencia al concepto de Artizón, a los elementos que le brindan el soporte teórico, así como a las causas que justifican su utilización; de la misma forma, se detallarán las técnicas y procedimientos que se seguirán para su uso, así como los resultados cuantitativos alcanzados a partir del uso de los Artizones. Término al cual se le ha dado una connotación integral, que contempla, tanto la utilización del material didáctico, como la dinámica grupal creada por el maestro facilitador, para lograr una adecuada participación de los alumnos; de la misma forma, el concepto Artizón concibe las estrategias empleadas, seleccionando para ello las técnicas individuales y/o grupales más convenientes, para lograr un verdadero aprendizaje significativo.
%%%%%%%%%%%%1701%%%%%%%%%%%
\subsection{\sffamily Uso de Categorías en la Resolución y Elaboración de Problemas tipo PISA {\footnotesize (, )}} \label{reg-1701} \index{Benítez Mariño Eloísa}
\noindent {\bfseries Eloísa  Benítez Mariño}, {\tt elobenitez@uv.mx}  {\slshape ( Universidad Veracruz (UV)  Facultad de Matemáticas )}\\
\noindent {\it  Coautor: José Rigoberto Gabriel Argüelles        }\\
\noindent  En esta plática se presentan otras maneras de mirar el aprendizaje del proceso de la competencia de categorizar como un integrante de la Metodología de la Investigación, una investigación que he estado desarrollando. Uno de mis resultados  de investigación, me facilitó el acceso a elaborar una experiencia de aprendizaje que me permitiera continuar con mi indagación sobre categorías, dado que éstas adquieren importancia cuando contribuyen en la elaboración de instrumentos que son utilizados por la investigación en las diferentes áreas científicas, en particular, en la evaluación educativa. 
%%%%%%%%%%%%942%%%%%%%%%%%
\subsection{\sffamily Departamentalización en los procesos de enseñanza y  evaluación en la Academia de Matemáticas de la Universidad Politécnica de San Luis Potosí {\footnotesize (, )}} \label{reg-942} \index{González Salas Javier Salvador}
\noindent {\bfseries Javier Salvador González Salas}, {\tt jsgs100573@hotmail.com}  {\slshape (Universidad Politécnica de San Luis Potosí (UPSLP) Academia de Matemáticas)}\\
\noindent {\it  Coautor: Cynthia Berenice Zapata Ramos        }\\
\noindent  Para adquirir un nuevo conocimiento existe la necesidad de tener ciertos conocimientos previos y habilidades específicas, por lo tanto es importante hacer esfuerzos para que el  estudiante sea capaz de adquirir ciertos estándares en conocimiento y habilidades. Para establecer dichos estándares, cada Institución elabora sus propios procesos para sus respectivos departamentos, academias y/o facultades.  Así pues para evaluar un estándar de habilidad y/o conocimiento en el estudiante, la gran mayoría de las Instituciones utilizan exámenes departamentales.  En este trabajo se presenta una descripción de cómo se departamentalizan los procesos de enseñanza y evaluación en el área de matemáticas de la Universidad Politécnica de San Luis Potosí  y las ventajas que ha ofrecido su implementación. Se muestran resultados de los promedios de aprobación y el número total de estudiantes atendidos desde el inicio de Nuestra Universidad. 
%%%%%%%%%%%%1789%%%%%%%%%%%
\subsection{\sffamily Una experiencia remedial universitaria {\footnotesize (, )}} \label{reg-1789} \index{Salazar Antúnez Marina}
\noindent {\bfseries Marina  Salazar Antúnez}, {\tt msalazar@correo.azc.uam.mx}  {\slshape (Universidad Autónoma Metropolitana - Azcapotzalco (UAM-A) Ciencias Básicas)}\\
          \noindent  UNA EXPERIENCIA REMEDIAL UNIVERITRIA Ante los altos índices de reprobación que se presentan en los primeros cursos de matemáticas para los estudiantes de ingeniería, la UAM-Azcapotzalco, a través de la División de Ciencias Básicas e Ingeniería, implementó un Programa de Nivelación Académica que arrancó en el trimestre de otoño del 2008. Esta plática pretende compartir la experiencia de dicho programa presentando su aplicación, su desarrollo y su evaluación a cuatro años de su inicio. 
%%%%%%%%%%%%1610%%%%%%%%%%%
\subsection{\sffamily Ecuaciones diferenciales y estrategias didácticas. {\footnotesize (, )}} \label{reg-1610} \index{PÉREZ FLORES RAFAEL}
\noindent {\bfseries RAFAEL  PÉREZ FLORES}, {\tt pfr@correo.azc.uam.mx}  {\slshape (UNIVERSIDAD AUTÓNOMA METROPOLITANA (UAM))}\\
\noindent {\it  Coautores: ERNESTO  ESPINOSA HERRERA, CARLOS ANTONIO ULÍN JIMÉNEZ      }\\
\noindent Se describe en la ponencia una experiencia educativa en la que están tomando parte estudiantes de diferentes programas de ingeniería de la Universidad Autónoma Metropolitana, Unidad Azcapotzalco. Dicha experiencia forma parte de un proyecto de investigación sobre el diseño y la aplicación de determinadas estrategias didácticas para el proceso de enseñanza de matemáticas. En la experiencia educativa que se describe, las estrategias didácticas en cuestión se diseñaron con el propósito de coadyuvar con el aprendizaje de los métodos para resolver ecuaciones diferenciales y sus aplicaciones, aprendizaje entendido como desarrollo de procesos de pensamiento. Se trata del estudio del desarrollo de procesos de pensamiento inductivo como un elemento del razonamiento lógico implícito en los problemas de aplicación de ecuaciones diferenciales. En esta experiencia educativa, las estrategias didácticas han tenido como propósito específico: guiar el pensamiento del estudiante, desde información particular (hechos, ejemplos, procesos de pensamiento con particularidades) hasta información general (conceptos, técnicas para resolver ecuaciones, procesos de pensamiento con generalidades). En otras palabras, se trata de proporcionar al estudiante información asequible a su intelecto para facilitar la vinculación de ésta con generalidades, es decir, coadyuvar con la comprensión. Como referentes teóricos se han tomado en consideración a dos importantes exponentes de posturas constructivistas: Ausubel y Bruner. Dentro de la teoría del Aprendizaje Significativo de Ausubel, un Aprendizaje Supraordinado representa un razonamiento inductivo, partir de información particular para llegar a información general. Para Bruner, un Aprendizaje por Descubrimiento inductivo es justo la adquisición y ordenación de datos para poder llegar u obtener nuevas categorías, conceptos o generalizaciones. Tal como lo señalan diversas investigaciones, iniciar los procesos de aprendizaje promoviendo procesos de pensamiento inductivos permite aprendizajes de calidad. La experiencia educativa en cuestión se está realizando actualmente en el curso Ecuaciones diferenciales Ordinarias (Edo) para las licenciaturas de ingeniería, en el cual los temas a tratar son: (1) Conceptos básicos, (2) Edo de primer orden, (3) Aplicaciones de Edo de primer orden, (4) Edo lineales de segundo orden y (5) Aplicaciones de Edo lineales de orden 2 con coeficientes constantes. Un problema que se presenta en este curso, al ser impartido en este orden, es el bajo nivel de aprendizaje que logran los alumnos en los primeros temas y concretamente en los temas (2) Edo de primer orden y (3) Aplicaciones de Edo de primer orden. Este deficiente aprendizaje se refleja en un bajo porcentaje de aprobación en la primera evaluación parcial y se mantiene en la evaluación global del curso. Como posibles causas de esta problemática pensamos en las siguientes: la escasa o pobre motivación que reciben y logran los alumnos en las primeras clases del curso, al ser bombardeados con definiciones áridas y sin mucho significado con respecto a sus pretensiones como estudiantes de ingeniería; así como también, el escaso tiempo que se dedica al tema de las aplicaciones y que lleva al alumno a memorizar las soluciones generales de cada tipo de problema, para luego sustituir valores y convertir un problema de ecuaciones diferenciales en un ejercicio netamente numérico. Con el objetivo de mejorar el aprendizaje de los alumnos en esta parte del curso, hemos pensado en lo siguiente: un cambio en el orden de estos temas podría coadyuvar en el logro de una mayor motivación al inicio del curso y un mejor aprendizaje de las aplicaciones. Aún más, entendiendo y saboreando las aplicaciones, podría aumentar en ellos la motivación para aprender a resolver ecuaciones diferenciales. En esta dirección, estamos experimentando la siguiente estrategia didáctica. Cambiar el orden de los tres primeros temas y cubrirlos de la siguiente manera.  Recordar a la derivada como razón de cambio y particularmente como rapidez de cambio. Modelar problemas de aplicación. Crecimiento de poblaciones (modelo Malthusiano). Decaimiento radioactivo. Ley de enfriamiento de Newton. Caída libre y no-libre con fricción proporcional a la rapidez. Mezclas y concentraciones. Cubrir el tema 1 de conceptos básicos. Cubrir los temas 2 y 3 mezclados. Aprender a resolver ecuaciones diferenciales y luego dar solución a los problemas de valores iniciales obtenidos en las modelaciones realizadas anteriormente. B. Resolver los problemas en general y no problemas numéricos. Dejar como tarea el estudio de los ejemplos numéricos resueltos en el libro de texto, para luego resolver los ejercicios propuestos en el mismo. Como se puede apreciar, el cambio en el orden de impartir los temas del curso Ecuaciones Diferenciales Ordinarias, intenta promover la realización de procesos de pensamiento inductivos. Pretendemos en esta conferencia, dar a conocer los resultados que se obtengan de esta experiencia educativa.   REFERENCIAS: AUSUBEL, D. (1976): Psicología educativa. Un punto de   vista cognoscitivo, México, Trillas. AUSUBEL, D. P.; NOVAK, J.D. y HANESIAN, H. (1988):   Psicología de la educación, México, Trillas. BRUNER, J. (1988): Desarrollo cognitivo y educación,   Madrid, Morata. BRUNER, J. (1997): La educación puerta de la cultura,   Madrid, Visor.
%%%%%%%%%%%%1424%%%%%%%%%%%
\subsection{\sffamily La velocidad instantánea y el  vector tangente desde una perspectiva de  física matemática {\footnotesize (, )}} \label{reg-1424} \index{Blancarte Suarez Herminio}
\noindent {\bfseries Herminio  Blancarte Suarez}, {\tt herbs@uaq.mx}  {\slshape (universidad autónoma de Querétaro ( uaq),facultad de ingenieria,licenciatura en matemáticas aplicadas)}\\
          \noindent Al tratar de abordar los temas del cálculo vectorial de su fuente histórica  natural como lo fue la mecánica clásica. Se tiene la posibilidad  de contextualizar  el proceso de enseñanza-aprendizaje al proveer de los ejemplos  naturales que dieron origen a conceptos y definiciones  del cálculo vectorial. La presente plática representa un ejemplo de este tipo de propuesta didáctica
%%%%%%%%%%%%1220%%%%%%%%%%%
\subsection{\sffamily Evaluación de Problemas de Cálculo desde la Perspectiva de la Competencia Matemática {\footnotesize (, )}} \label{reg-1220} \index{Vera Soria Francisco}
\noindent {\bfseries Francisco  Vera Soria}, {\tt fveraso@hotmail.com}  {\slshape (Departamento de Matemáticas Centro Universitario de Ciencias Exactas e Ingenierías Universidad de Guadalajara (U de G))}\\
          \noindent El presente trabajo reporta avances del proyecto de investigación sobre la evaluación de la competencia matemática en la resolución de problemas de cálculo, en el contexto del proceso de evaluación departamental en el Centro Universitario de Ciencias Exactas e Ingenierías CUCEI de la Universidad de Guadalajara. Es referida la competencia matemática, desde el conocimiento matemático como disciplina y desde una matemática escolar. En donde la evaluación de la competencia matemática es entendida como: traducir en juicios de valor la actividad matemática que permite emplear diversos elementos del hacer matemático para resolver nuevas situaciones problema.
