\section{De joven a joven}

Leonardo Ignacio Martínez Sandoval
Curiosidades de números primos

Leonardo Ignacio Martínez Sandoval
Evariste Galois: Un revolucionario, un matemático

Manuel Alejandro Juárez Camacho
Lo raro del infinito y otras curiosidades en mate

Luis Miguel García Velázquez
Buscando simetrías (y otras estructuras escondidas).

Susana Patiño Espinosa

Gasde Augusto Hunedy López
El misterio de la casa roja y la gráfica delatora

Denae Ventura
Una Introducción a las Coloraciones en Teoría de Gráficas

Ilán A. Goldfeder
Historia de una batalla naval argumentada matemáticamente

Anayanzi Delia Martínez Hernández
Las matemáticas en el reloj y los mensajes secretos

Micael Toledo

Iván Ongay Valverde
De infinitos a INFINITOS

María Cristina Cid Zepeda
Conociendo a los conjuntos fractales.

Rodrigo Jesús Hernández Gutiérrez
¿Cuántas dimensiones?

Julián Fresán
¿Sabes contar?

Alma Violeta García López
¿Dónde trabajan los matemáticos?

Luis Antonio Ruiz López
¡A jugar!

Erick García Ramírez
Las paradojas del infinito.

Ana Victoria Ponce Bobadilla
¿Cómo enviar información confidencial?

José Pablo del Cueto Navarro
Lenguajes y códigos

Alejandra Ramos Rivera
Una definición "diferente" de una gráfica.

Samuel Estala Arias

Pedro Franco

Alma Rocío Sagaceta Mejía  (UAM)
Jugando Canicas en el Espacio

Jesús Daniel Arroyo Relión (ITAM)
¿Un mono puede escribir las obras completas de Shakespeare?
