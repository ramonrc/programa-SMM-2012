\section{De joven a joven}

Leonardo Ignacio Mart\'inez Sandoval\\
Curiosidades de n\'umeros primos\\

Leonardo Ignacio Mart\'inez Sandoval\\
Evariste Galois: Un revolucionario, un matem\'atico\\

Manuel Alejandro Ju\'arez Camacho\\
Lo raro del infinito y otras curiosidades en mate\\

Luis Miguel Garc\'ia Vel\'azquez\\
Buscando simetr\'ias (y otras estructuras escondidas).\\

Susana Pati\~no Espinosa\\

Gasde Augusto Hunedy L\'opez\\
El misterio de la casa roja y la gr\'afica delatora\\

Denae Ventura\\
Una Introducci\'on a las Coloraciones en Teor\'ia de Gr\'aficas\\

Il\'an A. Goldfeder\\
Historia de una batalla naval argumentada matem\'aticamente\\

Anayanzi Delia Mart\'inez Hern\'andez\\
Las matem\'aticas en el reloj y los mensajes secretos\\

Micael Toledo\\

Iv\'an Ongay Valverde\\
De infinitos a INFINITOS\\

Mar\'ia Cristina Cid Zepeda\\
Conociendo a los conjuntos fractales.\\

Rodrigo Jes\'us Hern\'andez Guti\'errez\\
?`Cu\'antas dimensiones?\\

Juli\'an Fres\'an\\
?`Sabes contar?\\

Alma Violeta Garc\'ia L\'opez\\
?`D\'onde trabajan los matem\'aticos?\\

Luis Antonio Ruiz L\'opez\\
!`A jugar!\\

Erick Garc\'ia Ram\'irez\\
Las paradojas del infinito.\\

Ana Victoria Ponce Bobadilla\\
?`C\'omo enviar informaci\'on confidencial?\\

Jos\'e Pablo del Cueto Navarro\\
Lenguajes y c\'odigos\\

Alejandra Ramos Rivera\\
Una definici\'on ``diferente'' de una gr\'afica.\\

Samuel Estala Arias\\

Pedro Franco\\

Alma Roc\'io Sagaceta Mej\'ia  (UAM)\\
Jugando Canicas en el Espacio\\

Jes\'us Daniel Arroyo Reli\'on (ITAM)\\
?`Un mono puede escribir las obras completas de Shakespeare?\\
